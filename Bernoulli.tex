\section{Bernouilli's principle}

Let us approach Eq. \ref{eq:Euler_momentum_w_vort} in a completely
different manner. Instead of getting rid of the gradient terms, let us
express as many terms as we can as gradients, for reasons that will
become clear soon. First, the gravitational acceleration is, by
definition, $\bfg = -\nabla\varphi$, where $\varphi $ is the
gravitational potential energy ($g z$ if the $z$ axis is the
vertical). Now, the pressure gradient over the density may be related
to the gradient of pressure over density, thus:
\[
\nabla
\left(
\frac{p}{\rho}
\right) =
\frac{ \nabla p }{\rho} -
\frac{ p }{\rho^2} \nabla \rho 
\]
Therefore,
\begin{equation}
  \label{eq:Bern_unsteady}
  \frac{\partial \bfu }{\partial t} +
  \nabla
  \left(
    \frac12 u^2 +
    \frac{p}{\rho} +
    \varphi
  \right)
  =
  \bfu \times\vort +
  \frac{ p }{\rho^2} \nabla \rho 
 \end{equation}

Now, in steady flow we would have
\[
\nabla
\left(
\frac12 u^2 +
\frac{p}{\rho} +
\varphi
\right)
=
  \bfu \times\vort -
 \frac{ p }{\rho^2} \nabla \rho  .
\]

The term involving the vector product of the vorticity and the
velocity is perpendicular to both. Hence, it vanishes upon a scalar
multiplication with $\bfu$:
\begin{equation*}
%  \label{eq:Bern_u_times}
  \bfu\cdot
  \nabla
  \left(
    \frac12 u^2 +
    \frac{p}{\rho} +
    \varphi
  \right)
  = 
 - \frac{ p }{\rho^2} \bfu\cdot \nabla \rho  .
\end{equation*}
 
In steady flow, the continuity equation is
\[
\bfu\cdot \nabla \rho  +
\rho \nabla \cdot \bfu = 0 ,
\]
and we may also write
\begin{equation}
  \label{eq:Bern_u_times}
  \bfu\cdot
  \nabla
  \left(
    \frac12 u^2 +
    \frac{p}{\rho} +
    \varphi
  \right)
  = 
  \frac{ p }{\rho} \divu.
\end{equation}
%
Hence, if the flow is incompressible (technically, divergence-free, as
explained in section \ref{sec:incompressibility}),
\[
\bfu\cdot
\nabla
\left(
\frac12 u^2 +
\frac{p}{\rho} +
\varphi
\right)
=
0 ,
\]
which states the fact that the quantity
\[
h = \frac12 u^2 + \frac{p}{\rho} + \varphi
\]
is constant along a given streamline. This result is known as
Bernoulli's principle, and applies only to ideal, steady,
incompressible flow. (There is a variant of it that applies to
unsteady flow, as we will see in section \ref{sec:gravity_waves}.)
%
This combination is called ``the head'' and is customarily used in
elementary applications of this result. Some of its direct
applications are: the Venturi effect (by which the pressure decreases
in zones with higher velocities), slow drainage of containers, syphons
\ldots

We will also see that in some cases, like in potential flow, the
velocity field may be found independently of the pressure. This
principle then yields the corresponding pressure from the velocity.

%any field satisfies
%\[
%\frac{\partial \bfu }{\partial t} +
%\]
