
\section{Exercises}

\begin{enumerate}
\item \label{ex:vector_identity}) Prove that
\[
\bfu \times ( \nabla\times \bfq) =
%\frac12
\nabla_\bfq (\bfu\cdot\bfq) - \bfu \cdot \nabla \bfq ,
\]
for any two vector fields. By $\nabla_\bfq $ it is meant that the
gradient is applied only on $\bfq$ (Feyman's subscript notation).
Hint: the curl of a field is associated with a tensor with elements:
\[
\overline{( \nabla\times\bfq )_{ij}} =
\left(
\frac{\partial q_j}{\partial x_i} -
\frac{\partial q_i}{\partial x_j}
\right) \qquad
\overline{( \nabla\times\bfq )_{ij}} =
\sum_k \epsilon_{ijk}  ( \nabla\times\bfq )_k .
\]
In the latter, $\epsilon$ is a rank-three tensor, the totally
antisymmetric tensor (also known as the Levi-Civita symbol).  If this
seems mistifying, all that it is expressing is that e.g.  $
(\nabla\times\bfq)_z= \partial q_x /\partial y - \partial q_y
/\partial x$, et c\ae tera.

This tensor appears also in the vector product:
\[
(\bfu\times\bfq)_i= \sum_{j,k} \epsilon_{ijk} u_j q_k .
\]
These two results make the demonstration quite simple. We will
also be using these expressions in section \ref{sec:}, when
we talk about rotations of a particle.

\end{enumerate}
