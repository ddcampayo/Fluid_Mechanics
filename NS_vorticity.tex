% To be merged onto NS.tex

Begining with our momentum equation for a Newtonian fluid undergoing
athermal incompressible flow, \ref{eq:NS_usual}.
\begin{equation*}
  \frac{\partial \bfu }{\partial t}
  + \bfu \nabla \bfu
  =
  - \nabla p 
  + \nu \nabla^2 \bfu
  + \bfg ,
\end{equation*}
(where ``$p$'' is actually ``p/\rho$, as customary in incompressible
flow), we may begin modifying the convective acceleration using Lamb's
identity \ref{eq:Lambs_identity}:
\begin{equation*}
  \frac{\partial \bfu }{\partial t}
  + \frac12 \nabla (u^2)
  -\bfu\times\vort
  =
  - \nabla p 
  + \nu \nabla^2 \bfu
  + \bfg .
\end{equation*}
The viscous term may be written as $\nabla^2 \bfu =
-\nabla\times\vort$, from the expression for the curl of a curl
\ref{eq:curl_of_curl} for incompressible flows. Taking the curl of
this expression, all the terms that are gradients of fields vanish,
and we get
\begin{equation*}
  \frac{\partial \vort }{\partial t}
  -\nabla\times (\bfu \times \vort )
  =
  -\nu  \nabla\times\nabla\times\vort .
\end{equation*}
Now, the curl of a vector product may be simplified as in Section
\ref{sec:Euler_vorticity}, $\nabla\times (\bfu \times \vort ) =
-\bfu\nabla\vort + \vort\nabla\bfu$ (again, valid for incompressible
flow). The double curl can again be cast as minus the Laplacian, since
the curl is always divergence-free.

Therefore,
\begin{equation*}
  \frac{\partial \vort }{\partial t} +
  \bfu\nabla\vort 
  =
  \vort\nabla\bfu
  +\nu  \nabla^2\vort .
\end{equation*}

Finally,
\begin{equation*}
  \frac{d \vort }{d t}
  =
  \vort\nabla\bfu
  +\nu  \nabla^2\vort .
\end{equation*}

This is clearly the Euler expression for the evolution of vorticity,
times an extra, diffusion term. The constant of diffusivity is $\nu$,
the same as velocity. It may seem that little is changed from that
situation. The reality is much different: it is actually the boundary
conditions which are usually sources of vorticity, which is then
carried onto the rest of the fluid by a combination of advecion and
diffusion. The ratio of these two processes depend of course on the
particular fluid (through $\nu$) and the flow features.
