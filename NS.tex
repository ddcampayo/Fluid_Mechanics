\section{Stress tensor}

Let us return again to our particle in order to analyze its movement
when stress forces are applied onto its walls. The pressure force may
be consider a special case of the latter, but stress forces may also
have a shear component.

Thus, the horizontal force will be given by, in part, by the
contributions due to the walls at the left, back, and bottom
\begin{equation}
  \label{eq:wall_shear_stress}
  \left. dF_x \right|_\text{l,bk,bm} =
  % - p         dy\, dz
    \tau_{xx} dy\, dz
    \tau_{yx} dx\, dz
    \tau_{zx} dx\, dy .
\end{equation}
%
The compression stress $\tau_{xx}$ is therefore quite similar to the
pressure (see (Eq \ref{eq:}), and in fact will be seen to include a
$-p$ term. The minus sign appears because normal stresses are
historically defined as positive if pointing outside the particle
(i.e. in the direction of the normal vector).
%
%where we include not only the pressure force, as before 
%and a compression stress $\tau_{xx}$ quite similar to it,
In addittion, two shear stress forces appear. One of them, $\tau_{yx}
dx\, dz$ is the horizontal shear force on the back wall ($dx\, dz$
actually equals $dx\, dy$, but it is clearer to write it this
way). Similarly, $\tau_{zx} dx\, dy $ is the horizontal stress force
at the bottom wall.

Notice the convention for labeling: $\tau_{ij}$ is the stress in the
$j$ direction on a face normal to axis $i$. I.e. the left face has the
same convention as the right one, the back as the front, the top as
the bottom.

%First, the normal stress is
%$\tau_{xx} \parallel \bfn $. From it,
%$\tau_{xy} \parallel \bfe_z\times \tau_{xx} $,
%$\tau_{xz} \parallel \tau_{xx} \times \bfe_y $, respecting the
%$(x,y,z)$ cyclic order. A quick way to realize this is by considering
%the top face, on which these three directions coincide with the
%Cartesian axes. The directions on other faces are found by rotating
%these directions accordingly. The right hand may be used for this
%purpose.

To get the whole horizontal force, the contributions from the other
three walls must be considered:
\[
\left. dF_x \right|_\text{r,ft,up} =
%- p         dy\, dz
  \tau_{xx}(x+dx,y   ,z   )  dy\, dz +
  \tau_{yx}(x   ,y+dy,z   )  dx\, dz +
  \tau_{zx}(x   ,y   ,z+dz)  dx\, dy .
\]
%This time, the sign convention is positive.

To get the total horizontal force due to stresses, the two contributions
must be substracted:
\[
dF_x =
\left( \tau_{xx}(x+dx,y   ,z   ) - \tau_{xx} \right) dy\, dz +
\left( \tau_{yx}(x+dx,y   ,z   ) - \tau_{yx} \right)  dx\, dz +
\left( \tau_{zx}(x+dx,y   ,z   ) - \tau_{zx} \right) dx\, dy .
\]
Notice that the substraction is in order: if the stresses are the same
on those faces, the resultant force will then be zero (those forces
may exert a net torque, but not a force).  In general, they will be
different on those faces, as as been made explitic on their arguments.

The choice is such that an variation in a given direction is assigned
a positive sign. This affects later results, in particular the fact
that the pressure appears with a negative sign.

Expanding in Taylor series, we get the following net horizontal force:
\[
 dF_x  =
%-\frac{\partial p      }{\partial x}  dx\, dy\, dz
% +
 \left(\frac{\partial\tau_{xx}}{\partial x}  dx\right)\, dy\, dz
+\left(\frac{\partial\tau_{yx}}{\partial y}  dy\right)\, dx\, dz
+\left(\frac{\partial\tau_{zx}}{\partial z}  dz\right)\, dx\, dy .
\]

The volumetric horizontal force is then,
\[
f_x  =
%-\frac{\partial p      }{\partial x}
 %+
 \frac{\partial\tau_{xx}}{\partial x}
+\frac{\partial\tau_{yx}}{\partial y}
+\frac{\partial\tau_{zx}}{\partial z} .
\]

With integer notation for Cartesian coordinates:
\[
 df_1  =
%-\frac{\partial p      }{\partial x_1}
 %+
 \sum_{j=1,2,3}
 \frac{\partial\tau_{j1}}{\partial x_j} .
\]


In general, for any component we will have:
\[
 df_i  =
%-\frac{\partial p      }{\partial x_i}
 %+
 \sum_{j=1,2,3}
  \frac{\partial\tau_{ji}}{\partial x_j} .
\]

In order to use vector notation, we have to introduce the
divergence of a tensor:
\[
\nabla \cdot \tau = \sum_{j=1,2,3}
  \frac{\partial}{\partial x_j}  \tau_{ji} ,
\]
 which results in a vector:
\[
 d\bff  =
%-\nabla p
 %+
 \nabla \cdot \tau .
\]
%
The tensor $\tau$ has components $\tau_{ij}$. There is also
the matrix notation, by which
\[
\nabla \cdot \tau = \nabla\tran  \tau ,
\]
where $\nabla\tran$ is a transposed (row) vector operator, multiplying
matrix $\tau$ from the left.

The resulting Navier-Stokes equation is then,
\[
\rho \frac{d\bfu }{dt} = \nabla \cdot \tau   + \rho \bfg .
\]



\subsection{Newtonian fluids}

The previous equation is still too general, and a connection between
stress and strain is still needed. Here we consider the case in which
there is a linear relationship between both, which involves the
coefficient of viscosity.

To begin with, let us consider a simple case in which a fluid is
confined between two planes. One of them moves sideways with a certain
speed $u_0$, while the other is kept fixed. After a certain transient,
some force is needed in order to keep this shearing. The simplest
expression is
\[
F= \mu A \frac{u_0}{L} .
\]
The force is proportional to the area and to the velocity difference
between the planes. It is also inversely proportional to their
separation, $L$ (this fact being the least obvious). Finally, a
constant of proportionality is given by $\mu$, the viscosity
coefficient, or simply ``the viscosity''. This constant may vary with
temperature, density, pressure, but the point with Newtonian fluids is
that it does not vary with the velocity field (or its
derivatives). Later, in section \ref{sec:}, this flow will be solved
as a solution of the Navier-Stokes equations, the Couette flow. There,
it will be shown that the velocity is everywhere in the direction of
the force exerted on the upper plane, let us call it $x$, and varies
linearly between the planes, in the $y$ direction. Therefore, the only
components of the strain rate tensor are $\epsilon_{xy} =
\epsilon_{yx} = u_0 / ( 2 L )$. We therefore have
\[
\tau_{xy} = \mu  \epsilon_{xy} .
\]

With these in mind, let us look for a general relationship between
$\tau$ and $\epsilon$. This is much easier if we go to the principal
strain axes. These are the coordinates on which the strain rate is
diagonal. Such coordinate system always exist, since the strain rate
tensor is symmetric. Notice that in these system strains are not due to
shear, only to dilations.

A simple example would be to consider the flow $u_x = u_0 y / L$
(again, Couette flow). In this case,
\[
\epsilon=
\begin{pmatrix}
  0           &   u_0/(2L)  \\
   u_0/(2L)   &   0
\end{pmatrix} .
\]

It is easy to find the two eigenvalues and associated eigenvectors of
this matrix:
\begin{align*}
  \lambda_1=  u_0/(2L) \qquad &\bfv_1=(1/\sqrt{2}) \begin{pmatrix} 1 &  1 \end{pmatrix}\tran \\
  \lambda_1= -u_0/(2L) \qquad &\bfv_2=(1/\sqrt{2})  \begin{pmatrix} 1 &  -1 \end{pmatrix}\tran \\
\end{align*}
Notice the first eigenvalue correnspond to a dilation along the $x=y$
diagonal, while the second is a compression along the $x=-y$ one.

The diagonal strain rate matrix is then:
\[
\tilde{\epsilon}=
\begin{pmatrix}
  u_0/(2L)   & 0 \\
  0          & - u_0/(2L)
\end{pmatrix} .
\]

It is crucial to realize that the stress tensor is also diagonal in
this coordinate system. Otherwise, a pure dilation in one of the
principal directions would cause a shear stress in some other
direction. This does not mean, however, that the two tensors are
simply proportional. Instead, we may posit, for the top-most element
\[
\tilde{\tau}_{xx}=
- p + 
C_1 \tilde{\epsilon}_{xx} +
C_2 \tilde{\epsilon}_{yy} +
C_3 \tilde{\epsilon}_{y} .
\]
We have added a $-p$ term that has to be there even when there is no
movement. This is needed, since a diagonal stress tensor $\tau=-p\eye$
produces the $-\nabla p$ term of hydrostatics (in Couette flow, $p$ is
constant, and this term is not needed.) The minus sign of $-p$ stems
from the fact that the pressure force points toward the interior of a
particle.

If isotropy is assumed, there should be no distinction between
traverse directions $y$ and $z$. Therefore, $C_2=C_3$, and
\[
\tilde{\tau}_{xx}=
- p + 
C_1 \tilde{\epsilon}_{xx} +
C_2 ( \tilde{\epsilon}_{yy} +  \tilde{\epsilon}_{y}  ) =
- p + 
K \tilde{\epsilon}_{xx} +
C_2 ( \tilde{\epsilon}_{xx} + \tilde{\epsilon}_{yy} +  \tilde{\epsilon}_{y}  ) ,
\]
where the constant $K=C_1-C_2$. Notice the $C_2$ term is the
divergence of the velocity. But, it is also the trace of the strain
tensor, a quantity which is invariant under change of basis. We can
now write:
\[
\tilde{\tau}_{xx}=
- p + C_2 \nabla\cdot \bfu 
K \tilde{\epsilon}_{xx} .
\]

There will be similar expressions for $\tilde{\tau}_{yy}$ and
$\tilde{\tau}_{zz}$, but in them the coefficients must be the same ---
otherwise isotropy will be violated. Taking all together,
\[
\tilde{\tau}=
K \tilde{\epsilon}
+ ( - p + C_2 \nabla\cdot \bfu ) \eye.
\]

Now, we may go back to the original Cartesian system and find
\[
\tau=
K \epsilon
+ ( - p + C_2 \nabla\cdot \bfu ) \eye.
\]
The stress tensor is then also symmetric, a fact that is required in
order the particle be torsion-free (remember the fact that rotations
have no stress associated.)

A comparison with our previous result reveals $K=2\mu$. The constant
$C_2$ is called, in the theory of elasticity ``the second Lam\'e
coefficient'', and receives the symbol $\lambda$ (it is also called
the ``second viscosity coefficient'', $\mu$ being the first.) Then,
\[
\tau=
2 \mu \epsilon + ( - p + \lambda \nabla\cdot \bfu ) \eye.
\]

To make this explicit, this means that diagonal terms have the form
\begin{equation}
  \label{eq:tau_diagonal}
  \tau_{ii}=
  2 \mu \epsilon_{ii}  - p + \lambda \nabla\cdot \bfu ,
\end{equation}
while off-diagonal terms are
\begin{equation}
  \label{eq:tau_off_diagonal}
  \tau_{ij}=  2 \mu \epsilon_{ij} \qquad j \ne i
\end{equation}


(Why not always work in the system of principal axes? The answer is
simple: principal axes vary from one particle to another, since they
are defined by local values of velocity derivatives. The Cartesian
coordinate system, or any such (cylindrical, polar\ldots) is the same
for all particles.)

The off-diagonal terms have a neat expression when the strain rate
tensor is written in term of velocity derivatives:
\[
\tau_{ij}=
\mu
\left(
\frac{\partial u_i}{\partial x_j} +
\frac{\partial u_j}{\partial x_i}
\right)
\]

The diagonal ones, however, are somewhat puzzling:
\[
\tau_{ii}=
- p +
2 \mu \frac{\partial u_i}{\partial x_i}  + \lambda \nabla\cdot \bfu .
\]
The pressure term is natural, but there are two extra dynamical terms.

Let us define a mechanical pressure as minus one-third times the trace
of the stress tensor:
\[
\bar{p}=
-\frac13 \Tr \tau  = 
 p
- \left( \frac23 \mu  + \lambda \right) \nabla\cdot \bfu .
\]

Defining volume pressure
\begin{equation}
  \label{eq:vol_visc_definition}
  \eta:=\frac23 \mu  + \lambda,
\end{equation}
we may write the mechanical pressure as
\[
  \bar{p}=
  p - \eta \nabla\cdot \bfu .
\]

The result is that the mechanical pressure, defined in such a way, is
different from the thermodynamic pressure in an incompressible fluid.
There are several ways out of this puzzling result. One of them is to
assume simply that the fluid is incompressible. This is of course
entirely correct, but would limit the applicability of the theory to
incompressible problems.

Another approach is to boldly assume $( 2/3 ) \mu+\lambda=0$. This
step was taken by Stokes, and defines a ``Stokesian fluid''. On the
other hand, there is no clear evidence of a real fluid that may
satisfy such a relationship. Indeed, the few measures of $\lambda$
have show positive values (while $\mu$, as should be clear, is always
positive). We should then keep in mind that in some flows when
compressibility is important, mechanical pressure may differ from
thermodynamical one. One such example is the attenuation of sound
waves, explained in Section \ref{sec:sound_waves_att}.

The Navier-Stokes equation for Newtonian liquids is finally:
\begin{equation}
  \label{eq:NS_Newtonian}
  \rho \frac{d\bfu }{dt} =
  - \nabla p +
  2 \nabla \cdot ( \mu \epsilon)
  + \nabla [ \lambda ( \divu ) ]
  + \rho \bfg .
\end{equation}


\subsubsection{Pure shear and compression}

Recalling the definition of the strain tensor $\epsilon$
in Eq. \ref{eq:epsilon_as_grad_u}, the viscous force may be written as
\begin{equation*}
  \bff_\mathrm{v}=
  \nabla \cdot \left[ \mu  \left( \nabla\bfu + \nabla\bfu\tran \right) \right] +
  \nabla \cdot \left[ \lambda (\divu) \eye \right] ,
\end{equation*}
where $\nabla \cdot \left[ A \eye \right] $ is just another way to
write $\nabla A$, for any scalar field $A$.


If the volume viscosity of Eq. \ref{eq:vol_visc_definition} is
introduced, the equation may be rearranged to read
\begin{equation}
  \label{eq:pure_stress_pure_compression}
  \bff_\mathrm{v}=
  \nabla \cdot \left[ \mu  \left(
      \nabla\bfu + \nabla\bfu\tran  - \frac23 (\divu)  \eye
    \right)
  \right] +
  \nabla \cdot \left[ \eta (\divu) \eye \right] .
\end{equation}
This expression is rather elegant, since the term multiplied by $\mu$
describes pure shear flow, with no compression effects, and the term
with $\eta$, pure compression.

This is easily demonstrated, since
\[
  \Tr  \nabla\bfu = \divu \qquad\implies\qquad \Tr \left( \nabla\bfu -\frac13 (\divu) \eye \right) = 0 .
\]
The same goes for $ \nabla\bfu\tran$, hence the $2/3$ factor.


\subsection{Particular instances}

We here consider instancies in which Eq. \ref{eq:NS_Newtonian} for
Newtonian liquid is further simplified.


\subsubsection{Athermal case}

It may happen that the viscosity coefficients do not vary, when the
variations on the other fields are not too great. In particular,
viscosity depends on temperature quite strongly, as is evident when
heating up oil. The temperature has its own equation, to be explained
in the next chapter. In this case, Eq. \ref{eq:NS_Newtonian} may be
written as
\begin{equation*}
  \rho \frac{d\bfu }{dt} =
  - \nabla p +
   2 \mu \nabla \cdot  \epsilon
  + \lambda \nabla ( \divu ) )
  + \rho \bfg .
\end{equation*}

Now, $\nabla \cdot  \nabla\bfu = \nabla^2 \bfu$, as can be demonstrated
since the $i$-th component of the resulting vector is
\[
(\nabla \cdot \nabla\bfu )_i =
\sum_{j=1,2,3} 
\frac{\partial}{\partial x_j} 
\frac{\partial u_i}{\partial x_j}
=
\sum_{j=1,2,3} 
\frac{\partial^2 }{\partial x_j^2} 
u_i = \nabla^2 u_i .
\]

However, $\nabla \cdot  \nabla\bfu\tran = \nabla  (\divu)$:
\[
(\nabla \cdot \nabla\bfu\tran )_i =
\sum_{j=1,2,3} 
\frac{\partial}{\partial x_j} 
\frac{\partial u_j}{\partial x_i}
=
\frac{\partial}{\partial x_i} 
\sum_{j=1,2,3} 
\frac{\partial u_j }{\partial x_j} 
= (\nabla  (\divu))_i .
\]

This means the NS equation can be written, in this limit,
as
\begin{equation}
  \label{eq:NS_const_viscs}
  \rho \frac{d\bfu }{dt} =
  - \nabla p +
   \mu \nabla^2  \bfu
  + ( \lambda  + \mu) \nabla ( \divu ) )
  + \rho \bfg .
\end{equation}





\subsubsection{Incompressible, athermal case}

If, in addition to having constant viscosity coefficients, the flow is
incompressible, the terms with $\divu$ in the previous section may be
neglected.

The final momentum equation for an incompressible, athermal fluid, is
then
\begin{equation}
  \label{eq:NS_usual}
  \rho \frac{d\bfu }{dt} =
  - \nabla p 
  + \mu \nabla^2 \bfu
  + \rho \bfg .
\end{equation}
This equation, in this particular form, is the beginning of a vast
ammount of research in physics and applied mathematics.





\section{Dimensionless variables: the Reynolds number}

For the simplest, athermal incompressible case, the term due to
viscosity
\[
\mu \nabla^2 \bfu = \mu \frac{u_0}{L^2} \nabla^{*2} \bfu^* ,
\]
where we cast the variables into reduced form, as explained in
section \ref{sec:Euler_adim}.

Recall that in order to arrive to \ref{eq:Euler_just_before_adim}, the
whole equation was multipled by $L/(\rho_0 u_0^2)$. If we do that to our
momentum equation, the result is
\[
\rho^* \frac{d\bfu^* }{dt^* } =
-  \nabla^* p^*
+  \rho^* \bfg^* +
\frac{\mu }{\rho_ 0 L u_0 } \nabla^{*2} \bfu^* .
\]

The term $\mu /( \rho_ 0 L u_0)$ must be dimensionless (as can be
easily checked). It then represents a reduced viscosity, and should be
taken as such: a number that defines whether viscosity is important or
not in a given context.

Historically, however, it is its inverse that has a name, the
Reynolds' number:
\begin{equation}
  \mathbf{Re}= \frac{\rho_ 0 L u_0 }{\mu }.
\end{equation}

This number is therefore large when viscosity is small, and small when
it is large. It may also be defined as the ratio of inertial forces and
viscous forces:
\begin{equation}
  \mathbf{Re}= \frac{\rho_ 0 u^2_0 }{\mu u_0 / L }.
\end{equation}
Indeed, in the numerator $\rho_ 0 u^2_0$ is the typical strength of the
pressure, and in the denominator $\mu u_0 / L $ is the typical
strength of viscous stress forces.

In many mathematical contexts, all dimensions are forfeited, and the
momentum equation is simply written as
\begin{equation}
  \label{eq:NS_usual_reduced}
  \frac{d\bfu }{dt} =
  - \nabla p 
  + \frac{1}{\mathbf{Re}} \nabla^2 \bfu
  + \bfg .
\end{equation}







\section{Vorticity}
\label{sec:NS_vort}

The momentum equation \ref{eq:NS_usual} can be treated by the method
in Sec. \ref{sec:NS_vort} in order to obtain an equation for the
vorticity and velocity alone.

Using the expresion for the curl of a curl \ref{eq:curl_of_curl},
\[
  \nabla^2 \bfu = \nabla (\divu) - \nabla\times\vort =  - \nabla\times\vort ,
\]
where we have supposed incompressibility in the second equality. This
means \ref{eq:NS_usual} can be cast as
\[
  \frac{\partial \bfu }{\partial t} +
  \frac12 \nabla u^2 - \bfu \times\vort =
  - \frac{1}{\rho} \nabla p 
  + \bfg + 
  - \nu \nabla\times\vort ,
\]
where we have used Lamb's identity  \label{eq:Lambs_identity}.


Now we may apply the curl operator to the whole equation, to get
\[
\frac{\partial \vort }{\partial t} -
\nabla\times(\bfu\times\vort) = 0 .
\]

In e.g. \cite{wiki:Vector_calculus_identities} we find an expression
for the curl of a vector product:
\[
  \nabla \times (\mathbf {a} \times \mathbf {b} ) =
  \mathbf {a} \ (\nabla \cdot \mathbf {b} )
  -\mathbf {b} \ (\nabla \cdot \mathbf {a} )
  +(\mathbf {b} \cdot \nabla )\mathbf {a}
  -(\mathbf {a} \cdot \nabla )\mathbf {b} .
\]
Hence,
\[
  \nabla \times (\bfu \times \vort ) =
  \bfu \ (\nabla \cdot \vort )
  -\vort \ (\nabla \cdot \bfu )
  +(\vort \cdot \nabla )\bfu
  -(\bfu \cdot \nabla )\vort.
\]
The first term is always zero, since the vorticity is divergence-free
(it being the curl of a field). The second is if the flow is
incompressible. In this case,
\[
  \frac{\partial \vort }{\partial t}
  + ( \bfu \cdot \nabla )\vort =  (\vort \cdot \nabla )\bfu .
\]
The left hand side is the advective derivative. Then:
\[
  \frac{d \vort }{d t}  =  (\vort \cdot \nabla )\bfu ,
\]
which means vorticity is ``almost'' conserved. The righ-hand side
represents an advection of velocity by vorticity, which sounds a bit
upside-down. This term may be neglected if the flow is slow (since it
involves a term quadratic in the velocity), to arrive at
$\frac{d \vort }{d t}\approx 0$. We will see, in Section \ref{sec:},
how


In any case, the equation involves only the velocity and its curl, the
pressure having been ``curled-away.'' Indeed, there is a class of
numerical methods, called ``vortex methods'' in which this
simplification is employed.  In many applications, however, its
usefulness is limited. For example, boundary conditions may be
difficult to define for these terms.

There is, however, an important consequence of this equation: if a
given flow is curl-free at some instant, it must \emph{remain} so
at every other time (both future, and past). This is because the equation
is first order in time, and a null right-hand side translates into
a null change. We will see later that when viscosity is introduced,
a diffusion term appears which is second order in space. This term
is, in many cases, responsible for the generation of vorticity.
In the clearest instance, the no-slip boundary condition close
to solid walls, by which the velocity must be zero there, creates
zones of vorticity generation. That boundary condition cannot be
enforced within the Euler framework, because it only features first order
spatial derivatives of the velocity.

