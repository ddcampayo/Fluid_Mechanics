% to be merged onto creeping.tex

If we try $\psi=r^2 \sin^2\theta$ we find
\[
  {\cal L} \psi = \left[ a(a-1) - 2 \right] r^{a-2} \sin^2\theta ,
\]
so the possible exponent values for ${\cal L} \psi = 0$ are $a=2$ and
$a=-1$ (this basically solves the potential flow problem.)

Notice that, once again, the outcome contains a function that is very
similar to the input. This means that, applying twice the differential
operator we will get
\[
  {\cal L}^2 \psi =
  \left[   a  (a-1) - 2 \right]
  \left[ (a-2)(a-3) - 2 \right]
  r^{a-4} \sin^2\theta ,
\]
so  the possible exponent values for ${\cal L}^2 \psi = 0$ are $a=2$ and
$a=-1$, as before, plus $a=4$ and $a=1$.

The solution we try will be a combination of all four posibilities --- but
the $a=4$ we can discard right away because it is too high a divergence. Let
us write
\[
  \psi = u_0 \left[ \frac12 r^2 + \frac{A}{r} + B r \right] \sin^2\theta .
\]
Then our boundary conditions at $r=R$ imply
\begin{align*}
  \left.\frac{\partial \psi}{\partial r}\right|_{r=R} = 0 &\implies
                                                            R-\frac{A}{R^2} + B  =0 \\
  \left.\frac{\partial \psi}{\partial \theta}\right|_{r=R} = 0 &\implies
                                                            \frac12 R^2-\frac{A}{R} + BR  =0 ,
\end{align*}
from which we may obtain $A$ and $B$, and finally write the solution
\[
  \psi = \frac12 u_0 R^2
  \left[\left(  \frac{r}{R} \right)^2
    + \frac12 \frac{R}{r} - \frac32 \frac{r}{R} \right] \sin^2\theta .
\]
The components of the velocity are readily found from
\ref{eq:u_from_psi_spherical},
\begin{equation*}
  \begin{split}
    u_r     &=  u_0  \left[
      1+ \frac12 \left(\frac{R}{r}\right)^3 - \frac32 \frac{R}{r}
    \right] \cos \theta \\
    u_\theta &=
    - u_0  \left[
      1
      - \frac14 \left(\frac{R}{r}\right)^3 - \frac34 \frac{R}{r}
    \right] \sin \theta
  \end{split}
\end{equation*}

The pressure is found from \ref{eq:}, and found to have a simple expression,
\[
  p=p_0 - \frac32 \frac{\mu u_0}{R} \left(\frac{R}{r}\right)^2
  \cos\theta ,
\]
where $p_0$ is the pressure far from the sphere. Notice this pressure
is harmonic, as discussed above. At the sphere surface,
\[
  p(r=R) = p_0 - \frac32 \frac{\mu u_0}{R} \cos\theta,
\]
with a high-pressure zone at the fore of the sphere (the point facing
the upstream direction, $\theta=\pi$), and a lower pressure at the aft
($\theta=0$). This clearly asymmetric pressure distribution results in
a net drag on the sphere
\[
  D_p=  - \int_S  p \cos\theta dA =
  -2\pi R^2 \int_0^\pi d\theta \sin\theta  p(\theta) \cos\theta =
  + 3 \pi \mu u_0 R \int_0^\pi d\theta \sin\theta  \cos^2\theta .
\]
The minus sign comes from the fact that we want the pressure upon the
sphere, which by reaction is opposite the pressure upon the fluid.
The last integral is immediate: the primitive is $-(1/3)\cos^3\theta$,
so the definite integral provides a $2/3$ factor. Finally,
\[
  D_p=  2 \pi \mu u_0 R .
\]
This agrees with our guess of \ref{eq:drag_sphere_guess}.

However, this drag force is due to pressure only. There will be an
additiona shear drag, which may be calculated from the shear stress.
The relevant component of the tensor is (see \cite{white1991viscous},
Eq. B17, p. 584):
\[
  \tau_{r\theta} = \mu \left(
    \frac1r
    \frac{\partial u_r}{\partial \theta} +
    r \frac{\partial }{\partial r} \left( \frac{u_\theta}{r}\right) 
  \right) .
\]

This can be computed to obtain
\[
  \tau_{r\theta} = 
  - \frac32 \frac{\mu u_0}{r} \left(\frac{R}{r}\right)^3 \sin\theta ,
\]
hence at the sphere surface,
\[
  \tau_{r\theta} (r=R) =  - \frac32 \frac{\mu u_0}{R}  \sin\theta .
\]
This looks very similar to the pressure, with a $\sin$ instead of a
$\cos$, which means the stress is maximal at the equator, which seems
reasonable. However, the contribution to the drag force is higher ---
twice higher, to be precise. Indeed,
\[
  D_\tau =  - \int_S  \tau_{r\theta} \sin\theta dA ,
\]
with a minus sign for the same reason as the pressure, and a $\sin$
because this is a shear stress, resulting in a force that is
tangential to the surface of the sphere.  Then
\[
  D_\tau =
  - 2\pi R^2 \int_0^\pi d\theta   \tau_{r\theta} \sin^2\theta  =
  3 \pi \mu u_0 R  \int_0^\pi d\theta \sin^3\theta .
\]
The primitive of $\sin^3\theta$ is, through
$\cos^2\theta=1-\sin^2\theta$, equal to
$(1/3) \cos^3\theta - \cos\theta$. The definite integral is then $4/3$, and
\[
  D_\tau = 4 \pi \mu u_0 R .
\]

The final drag force is the celebrated Stokes drag law: \index{Stokes drag law}
\[
  D_\tau = 6 \pi \mu u_0 R ,
\]
where $1/3$ of the drag is due to pressure and $2/3$ to shear stress.

Despite the shear having a contribution that is twice that two
pressure, notice that by writing 
\begin{align*}
  - p(r=R) + p_0 &= \bar{p} \cos\theta,  
  -\tau_{r\theta} (r=R) &= \bar{p} \sin\theta,  
\end{align*}
where $\bar{p} := - (3/2) \frac{\mu u_0}{R}$, it is apparent that the
normal force, due to pressure, and the tangential one, due to shear
stress, result from a decomposition of a force upon the sphere surface:
\[
  \bff := \frac32 \frac{\mu u_0}{R} \bhe_z .
\]
It is remarkable that this force is equal for every point on the
sphere.




FOR POTENTIAL.TEX:




In axisymmetric flow,
\[
  \divu =
  {1 \over r^2}{\partial \left( r^2 u_r \right) \over  \partial r} +
  {1 \over r\sin\theta}{\partial \over \partial  \theta} \left( u_\theta\sin\theta \right) ,
\]
but an equivalent expression is
\[
  \divu =
  {\partial \left( r^2 u_r \sin\theta \right) \over  \partial r} +
  {\partial \left( r u_\theta\sin\theta \right)  \over \partial  \theta}
\]

so continuity is trivially satisfied with the choice
\ref{eq:u_from_psi_spherical}.


From the curl in spherical coordinates,
\begin{align}
  u_r     &= \frac {1}{r\sin \theta }
            \frac{ \left( \partial A  \sin \theta \right) }{\partial \theta }  \\
  u_\theta &= -\frac1{r} \frac{\partial (rA) }{\partial r} 
\end{align}
and we find
\begin{equation}
%  \label{eq:u_from_psi_spherical}
  \begin{split}
    u_r     &=  \frac1{r^2 \sin\theta} \frac{\partial \psi}{\partial \theta} \\
    u_\theta &= -\frac1{r   \sin\theta} \frac{\partial \psi}{\partial r}
  \end{split}
\end{equation}



Notice $A=\psi / \rho $ also in spherical coordinates !