Let us find the work done on a fluid particle.

The First Law of thermodynamics tells us this change is due
to work (no heat exchange is considered here):
\[
dE=dW
\]

On the $x$ direction, this work will be given by the distance
travelled by the left wall in the $x$ direction times the pressure
force:
\[
dW_x (\mathrm{left})  = u_x \Delta t p A .
\]

This is work done on the particle (if $u_x$ is positive), hence its
sign. There will be a similar contribution from the right wall, so the
total work given by the $x$ direction will be
\[
dW_x = u_x \Delta t p A  -  u_x'  \Delta t p' A .
\]
By expanding in a Taylor series, this may be written as
\[
dW_x = - \frac{\partial p u_x}{\partial x} \Delta x A \Delta t.
\]

Now, adding the other three components we find
\[
\frac{dW}{dt} = - V \nabla\cdot (p\bfu) .
\]

The right hand side may be expanded
\[
\nabla\cdot (p\bfu) =
 p \nabla\cdot \bfu + 
 \bfu \cdot \nabla p .
\]

Recalling that, from the Euler equation \ref{eq:Euler} ,
\[
\nabla p  = \rho \bfg - \rho \frac{d \bfu}{dt} ,
\]
we may write
\[
\frac{1}{V} \frac{dW}{dt} =
-  p \nabla\cdot \bfu +
\rho  \bfu \cdot \frac{d \bfu}{dt}  -
\rho  \bfg \cdot  \bfu .
\]

Now, multiplying by $1/\rho$:
\[
\frac{1}{V} \frac{d \epsilon }{dt} =
 \bfu \cdot \frac{d \bfu}{dt}  -
 \bfg \cdot  \bfu 
-  \frac{p}{\rho} \nabla\cdot \bfu .
\]
Where we have defined the specific energy $\epsilon=E/M$.

This may be written as
\[
\frac{d}{dt} \left[
  \epsilon - \frac12 u^2  + \bfg \cdot \bfu
  \right] = -  \frac{p}{\rho} \nabla\cdot \bfu ,
\]
or
\[
\frac{d}{dt} \left[
  \epsilon - \frac12 u^2  - \varphi
  \right] = -  \frac{p}{\rho} \nabla\cdot \bfu ,
\]

This is clearly a law for
\[
e = \epsilon - \frac12 u^2  - \varphi ,
\]
where we define $e$, the specific internal energy as the total energy
minus kinetic energy, minus gravitational potential energy. In other words,
\[
\epsilon  = e  + \frac12 u^2  + \varphi .
\]

Notice also that in our steady Bernoulli principle,
Eq. \ref{eq:Bernoulli} we had
\[
\bfu \cdot
\nabla \left[
  \frac12 u^2 + \frac{p}{\rho} + \varphi
  \right] =  \frac{p}{\rho} \nabla \cdot \bfu ,
\]
while in the steady state,
\[
\bfu \cdot \nabla e = -  \frac{p}{\rho} \nabla\cdot \bfu.
\]
Combining both we have the steady state Bernoulli equation for
a compressible fluid:
\[
\bfu \cdot \nabla \left[ e + \frac12 u^2 + \frac{p}{\rho} + \varphi
  \right] = 0 ,
\]

which tells us the combination $ e + \frac12 u^2 + \frac{p}{\rho} +
\varphi = \epsilon + \frac{p}{\rho} $ is constant along streamlines.
Notice that, by the definition of enthalpy,
\[
H= E + P V ,
\]
the specific enthalpy is
\[
h=H/M = e + p / \rho ,
\]
so the Bernoulli principle claims that the total specific enthalpy
($h'=h + \frac12 u^2 + \varphi$) is constant along streamlines.

We may also find an equation for the enthalpy. In Eq. \ref{eq: }
above, the right hand side maybe changed using the continuity
equation:
\[
- \rho \nabla\cdot \bfu = \frac{d\rho}{dt} .
\]
which means
\[
-  \frac{p}{\rho} \nabla\cdot \bfu =
\frac{p}{\rho^2} \frac{d\rho}{dt}
\]

But
\[
\frac{d (p / \rho) }{dt} =
\frac{ 1 }{\rho} 
\frac{d p }{dt} -
\frac{ p }{\rho^2} 
\frac{d \rho }{dt}  .
\]
Hence,
\[
\frac{d ( e + p / \rho) }{dt} = \frac{ 1 }{\rho} \frac{d p }{dt} ,
\]
or
\[
\frac{d h }{dt} = \frac{ 1 }{\rho} \frac{d p }{dt} .
\]
Changes in the enthalpy are therefore tied to changes in pressure.


\subsection{Heat flux}

In the case that some heat flux is present, the First Law of
thermodynamics reads:
\[
dE=dW + dQ .
\]

The influx of heat to the particle, $dQ$, will enter our equations
from a vector heat flux $\bfq$, with units of energy/(area $\times$
time). For example, the heat influx due to transfer in the $x$ direction
will be
\[
dQ_x= \bfq dt dy dz - \bfq(x+dx) dt dy dz \approx
      - V \, dt \, \frac{\partial\bfq}{\partial x},
\]
where in the last approximation a Taylor expansion has been employed,
as should be customary by now.

Altogether, the total heat flux rate is
\[
\frac{dQ}{dt} = -V \nabla\cdot \bfq,
\]
and the final equation for the change of specific internal energy of
an ideal fluid is
\[
\rho \frac{de}{dt}  = -  p \nabla\cdot \bfu  - \nabla\cdot \bfq .
\]

The heat flux must be either due to an external source, or due to heat
diffusion. In the latter case, a well-known theory is Fourier's law, by
which the flow is due to a temperature gradient:
\[
\bfq=- k \nabla T ,
\]
where $k$ is the heat diffusion coefficient. The temperature does not
appear in the energy equation, but a common assumption is that
\[
e = c T ,
\]
where $c$ is the specific heat, taken as constant. Then,
\[
\rho  c \frac{d T }{dt}  = - p \nabla\cdot \bfu  + k \nabla^2 T .
\]

For an incompressible fluid,
\[
 \frac{d T }{dt}  = \frac{ k }{ c \rho} \nabla^2 T  = \alpha \nabla^2 T ,
\]
where $\alpha = k / (c \rho)$, a constant if $\rho$ is. This is the
convective Fourier heat equation (it is not the most common Fourier
law, due to the derivative being convective, and not just partial.)

