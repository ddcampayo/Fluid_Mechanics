\section{Dimensionless variables}
\label{sec:Euler_adim}

A procedure to gain insight into a physical problem is to try to cast
the different magnitudes into dimensionless (or, ``reduced'')
ones. For example, if there is a relevant length scale $L$, all
lengths may be rescaled according to it:
\[
x^*=\frac{x}{L} \quad
y^*=\frac{y}{L} \quad
z^*=\frac{z}{L} ,
\]
where an asterisk marks a dimensionless magnitude. We can also write
it in vector notation: $\bfr^* = \bfr / L$.

As an example, in some problems this is the only relevant scale, and
the movement is driven by gravity, whose accelaration is $g$. In such
cases, the time scale will be given by the only combination of $L$ and
$g$:
\[
t_0 \sim \sqrt{\frac{ L }{ g }} .
\]
This is actually the correct result for the period of a simple
pendulum, but for a numerical factor of $2\pi$ with no dimensions:
$T=2\pi t_0$. No equations have been solved (or even written down) in
order to arrive to this result. Notice also that for larger
displacements of the pendulum, the amplitude of the motion is another
length, which complicates the analysis.

In many fluid problems there is a well-defined velocity $u_0$ that
sets the velocity values (e.g. the upstream velocity in flows around
objects). If this is the case,
\[
\bfu^* = \frac{\bfu}{u_0} \qquad
t^*=\frac{t}{L/u_0} ,
\]
so the velocity and length set the time scale. If we apply this to
our Euler equation,
\[
\rho \frac{d\bfu }{dt} =
- \nabla p 
+ \rho \bfg  \qquad
\rho \frac{u_0}{L/u_0} \frac{d\bfu^* }{dt^* } =
- \nabla p 
+ \rho \bfg .
\]

Notice that the $\nabla$ operator can also be cast into dimensionless
form. For example, its $x$ component is
\[
\nabla_x = \frac{\partial}{\partial x} = \frac{\partial}{\partial (x^* L) } =
\frac{ 1 }{ L } \frac{\partial}{\partial x^* },
\]
so we may define
\[
\nabla^* = \frac{ 1 }{ L } \nabla.
\]

The Euler equation then reads,
\[
\rho \frac{u_0}{L/u_0} \frac{d\bfu^* }{dt^* } =
- \frac{ 1 }{ L } \nabla^* p 
+ \rho \bfg .
\]

Usually, a reference value $\rho_0$ for the density is often known, so
that $\rho^* = \rho / \rho_0$, and
\[
\rho_0 \rho^* \frac{u_0}{L/u_0} \frac{d\bfu^* }{dt^* } =
- \frac{ 1 }{ L } \nabla^* p 
+ \rho \bfg .
\]


Now, multiplying throughout by $L/(\rho_0 u_0^2)$, and supposing for simplicity
that the density is constant,
\begin{equation}
  \label{eq:Euler_just_before_adim}
\rho^*
\frac{d\bfu^* }{dt^* } =
-  \nabla^* \frac{ p }{ \rho_0 u_0^2 } 
+  \frac{ L }{ u_0^2 } \bfg \qquad \implies \qquad
\rho^* \frac{d\bfu^* }{dt^* } =
-  \nabla^* p^*
+  \rho^* \bfg^* 
\end{equation}
We have therefore found that the dimensionless pressure and gravity
acceleration are given by
\[
p^*= \frac{ p }{ \rho_0 u_0^2 } \qquad
\bfg^* = \frac{ L }{ u_0^2 } \bfg .
\]

The reduced pressure was to be expected, given the Bernoulli
expressions involving $\rho u^2$. The reduced gravity is directly
related to Froude's number, which is historically defined as
\[
\mathrm{Fr}=\frac{ \sqrt{gL}}{u_0}.
\]
Therefore, $ \bfg^* = \mathrm{Fr}^2 (\bfg/g) $, where vector $(\bfg/g)
$ is the unit vector pointing in whichever direction the gravity
points to in our problem (usually, $-y$ or $-z$.)


It is easy to check that the continuity equation can likewise be
cast into reduced form:
\[
\frac{d\rho^* }{dt^* }+
 ( \nabla^* \bfu^*) = 0 .
\]
