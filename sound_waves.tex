\label{sec:sound_waves}

The existence of pressure and density waves, as solution of the Euler
equations, is established in this section.

Our set of equations includes the Euler momentum equation, neglecting
gravity
\[
\frac{d \bfu }{\partial t}= - \frac{1}{\rho} \nabla p  .
\]

Continuity is also important: since we are expecting density waves,
this quantity should not be considered to be constant. The full
equation is therefore:
\[
\nabla \cdot \rho \bfu + \frac{\partial \rho}{\partial t} =0
\]

Finally, in order to close the equations, we need some relationship
between variations in the density and variations in the pressure. This
comes from thermodynamics. Indeed, the reference pressure and density
are tied by an equation of state
\[
p=p(\rho,T) .\] For example, at about $T_0=\SI{300}{\kelvin}$, and a
pressure of $p_0=\SI{1e3}{\hecto\pascal}$, we know
$\rho_0\approx\SI{1.225}{\kilo\gram\per\cubed\meter}$, since those are
so-called standard conditions.

\section{Linearization}

Now, we will assume small fluctuations about constant backgrounds. I.e:
\begin{align}
  \label{eq:sound_small_u}
  \mathbf{u} & \approx \mathbf{u} \\
  \label{eq:sound_small_rho}
  \rho       & \approx \rho_0 + \rho' \\
  \label{eq:sound_small_p}
  p          & \approx p_0+ p' \approx p_0+ \kappa \rho'
\end{align}
Equation \ref{eq:sound_small_u} means the velocity is to be considered
small in some sense, but there is no average drift on top of which our
waves may travel. Such a drift may be present if there are currents
(water) or winds (air), and it is instructive to derive its effect. We
will neglect it here.

In Equation \ref{eq:sound_small_p} implies that a small increase in
density should cause a small increase in the pressure. It is hard to
assume otherwise, except when close to a phase
transition. Mathematically, this is a Taylor expansion:
\[
p \approx
p_0 +
\left. \frac{d p}{d\rho} \right|_{\rho_0} (\rho-\rho_0)
\qquad\to\qquad
p' \approx \left. \frac{d p}{d\rho} \right|_{\rho_0} \rho' .
\]
The proportionality is given by $\kappa$, defined as
\begin{equation}
  \label{eq:sound_kappa}
  \kappa =  \left. \frac{d p}{d\rho} \right|_{\rho_0} 
\end{equation}

Then, neglecting second and higher order perturbation terms we may
obtain linearized equations, involving only the velocity and the
pressure:
\begin{align}
  \label{eq:sound_lin_cont}
  \frac{\partial \bfu }{\partial t} & = - \frac{1}{\rho_0} \nabla p' \\
  %
  \rho_0 \kappa \nabla \cdot \mathbf{\rho u} +
  \frac{\partial p'}{\partial t} =0 
\end{align}
The kinematic viscosity is defined as $ \nu=\mu/\rho_0$.%
\index{kinematic viscosity}
Notice the convective part of the material derivatives is neglected,
since it is of higher order.

If we differentiate with respect to space (by applying $ \nabla$) in
the first equation, and with respect to time in the second, we can
substract the two, and eliminate the velocity term. 
The following equation for the pressure results:
\[
\frac{\partial^2 p' }{\partial t^2 }  = \kappa \nabla^2 p' ,
\]
which is the wave equation. Its solutions are waves, and their speed,
the speed of sound, is $c^2 = \kappa $. Our first impulse should be to
check whether our prediction of the speed of sound matches the known
value for some standard situations. By far, the best known situation
is the speed of sound in air in standar conditions. Since air is, to
a good approximation, a perfect gas, then
\[
p V = n R T \qquad\to\qquad p  = \rho\frac{ R T}{m},
\]
where $m$ is the molar gas.

Then,
\[
\kappa =  \left. \frac{d p}{d\rho} \right|_{\rho_0} =
 \frac{ R T_0}{m} = \frac{ p_0 }{ \rho_0 } .
\]
The last equation saves us from checking up the value of $R$ and the
molar mass of air (an effective value, since the air is a mixture of
gases.)

Our numerical result would then be
\[
c= \sqrt{\frac{ p_0 }{ \rho_0 }} =
\sqrt{\frac%
  { \SI{1e3}{\hecto\pascal} }%
  {\SI{1.2}{\kilo\gram\per\cubed\meter}}} =
\SI{290}{\meter\per\second} .
\]

This number, while not terribly wrong, is not equal to the speed of
sound in standard conditions, which is about
$\SI{343}{\meter\per\second}$. (A mnemonic rule is that it takes about
three seconds for the air to cover one kilometer --- hence the
practice of counting the time in seconds between lighning flash and
thunder sound, and divide by three to estimate the distance to a
lightning in a storm.)

The mistake implicit in our calculations, which was present in the
original derivation by Isaac Newton, was later corrected as our
understanding in thermodynamics increased throughout the XIX
century. The error is to perform the derivative $d p' / d\rho'$ at
fixed temperature. We know now that the speed of sound sets the limit
at which a thermodynamic process may be considered isothermal. For
sound waves, the process is better modeled as
adiabatic\index{adiabatic}.

In adiabatic processes, each fluid particle is unable to tranfer
energy to its surroundings, since the wave phenomenon is too
fast. Without going into details (which may be found at any standard
book on thermodynamics), the well-known law for an adiabatic process is
\[
p V^\gamma = \mathrm{ct} ,
\]
while in our previous, isothermal derivation, we had assumed $ p V =
\mathrm{ct} $. The exponent $\gamma$ is the famous adiabatic exponent,
which depends on the ideal gas being monoatomic or diatomic (in the
standard temperature range, at least). The air is really neither,
being a mixture, but most of its components (N$_2$ and $O_2$) are
diatomic, so a good approximation to $\gamma$ is $7/5$, the diatomic
value.

We can manipulate the adiabatic relationship this way:
\[
p V^\gamma = \mathrm{ct} \qquad\to\qquad
p V^\gamma = \mathrm{ct}_2 m^\gamma \qquad\to\qquad
p = \mathrm{ct}_2 \rho^\gamma ,
\]
where $\mathrm{ct}_2$ is some other constant, and we introduce
$m^\gamma$ because the mass of a particle, $m$, is constant (as we
will ellaborate again later, see e.g. Section \ref{sec:continuity3} .)
By the shortcut of applying logarithms we readily find our $\kappa$:
\[
\log p = \mathrm{ct}_3 + \gamma \log \rho  \qquad\to\qquad
\frac{1}{ p }
\frac{d p}{d\rho}=  \gamma \frac{1}{\rho} ,
\]
i.e.
\[
\kappa = \gamma \frac{p_0}{\rho_0} .
\]
Our new numerical result is the same as before, but with
an extra $\sqrt{7/5}$ factor, a $14\%$ increase approximately.
Finally:
\[
c= \sqrt{\frac{\gamma p_0 }{ \rho_0 }} =
\sqrt{\frac%
  { (7/5)  \SI{1e3}{\hecto\pascal} }%
  {\SI{1.2}{\kilo\gram\per\cubed\meter}}} =
\SI{330}{\meter\per\second} ,
\]
a value much closer to the real value of $\SI{343}{\meter\per\second} $.

Let us finish by analyzing a simple sound wave. Our wave is harmonic:
it has a well defined frequency $f$, and therefore angular frequency
$\omega=2\pi f$. Of course, the word ``harmonic'' stems from our
knowlodge of sound waves, hence ``harmonic analysis'', and so on (but,
the equivalent concept in optics is ``monochromatic''.) These sort of
waves sound ``pure'' (also, quite dull) and may be obtained from
tuning forks or, in our days, synthetisers. It is also planar, since
it depends only on $x$ --- no such waves exist, but this is a rather
good approximation for small regions far away from the sound source.
Its mathematical expression is
\[
\rho = \rho_0 +  \rho_A \cos[ k ( x  - \omega/k  t) ]
= \rho_0 \rho_A \cos(k x  - \omega t) .
\]
It is traveling since, as shown in the first equality, it is a
function of the form $f(x-ct)$. Also, $c= \omega/k$, which fixes the
relationship $\omega = c k$. Since $\omega$, the angular frequency, is
proportional to $k$, the wave number, this wave is non-dispersive: the
speed of sound is the same for all wave numbers. The wave number is
related to the wave-length, $k=2\pi/\lambda$, therefore $f \lambda =
c$. The relationship between the length of a wave and its frequency is
of course very relevant for musical instrument makers and architecture
of buildings with musical interest (concert halls, recording studios.)

The amplitude of the density modulation is $\rho_A$, and $\rho_A \ll
\rho_0$ since modulations are small.

The corresponding pressure modulation is
\[
p =  p_0 + \rho_A c^2  \cos(k x  - \omega t) .
\]
The number $ \rho_A c^2 $ is much ``larger'' than $\rho_A$, but that
should not bother us: a large density modulation of, say $\rho_A=
{\SI{0.1}{\kilo\gram\per\cubed\meter}}$ (which means a modulation of
the air density of approximately $10\%$ about its unperturbed value)
translates into about ${\SI{140}{\hecto\pascal}}$, which is a $14\%$
pressure modulation, more or less.

Finally, the velocity may be recovered from \ref{eq:sound_lin_cont}. The
solution to it is
\[
\bfu = u_0  \cos(k x  - \omega t) \bfe_x .
\]
It is therefore a purely longitudinal wave, since the particles move
in the direction of wave propagation, $x$ (at variance with planar
light waves in vacuum, which is purely transversal).

The value of $u_0$ is set by the requirement that $\bfu$ is indeed a
solution:
\[
\omega \rho_A - \rho_0 u_0 k = 0 \qquad\to\qquad
u_0 = c \frac{\rho_A}{\rho_0} .
\]

The velocity of the particles is therefore related to the speed of
sound $c$, but lower than it, since the ratio $\rho_A / \rho_0$ (of
about $0.1$ in the example above) must be small for our approximations
to remain valid. Also, all three fields, pressure, density, and
velocity are in phase. The speed is moving to the right, and the zones
with high pressure and density move to the right, while the ones with
low pressure and density move to the left. If the waved moved to the
left, the sign of $u_0$ would change, and the situation would be the
opposite.


\section{Mach's number and compressibility}
