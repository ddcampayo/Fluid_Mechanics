\section{Vorticity}
\label{sec:Euler_vorticity}

The momentum equation \ref{eq:Euler_momentum} is still quite daunting
due to its nonlinear term in the convective derivative.

In order to make progress, it was Lamb's idea in 1895 to use the
following identity for the vector product of any vector field and its
curl:
\begin{equation}
  \label{eq:Lambs_identity}
  \bfu \times ( \nabla\times \bfu) =
  \frac12 \nabla u^2 - \bfu (\divu )
\end{equation}
(in fact, the identity is somewhat more general, see exercise
\ref{ex:vector_identity}).

Introducing the name ``vorticity'' for the curl of the velocity field,
\[
\vort := \nabla\times\bfu ,
\]
we may write the momentum equation as
\begin{equation}
  \label{eq:Euler_momentum_w_vort}
  \frac{\partial \bfu }{\partial t} +
  \frac12 \nabla u^2 - \bfu \times\vort =
  - \frac{1}{\rho} \nabla p 
  + \bfg .
\end{equation}

Now, recalling that the curl of a gradient is always zero,
we may apply the curl operator to the whole equation.
If we assume constant density, then
\[
\frac{\partial \vort }{\partial t} -
\nabla\times(\bfu\times\vort) = 0 .
\]

In e.g. \cite{wiki:Vector_calculus_identities} we find an expression
for the curl of a vector product:
\[
  \nabla \times (\mathbf {a} \times \mathbf {b} ) =
  \mathbf {a} \ (\nabla \cdot \mathbf {b} )
  -\mathbf {b} \ (\nabla \cdot \mathbf {a} )
  +(\mathbf {b} \cdot \nabla )\mathbf {a}
  -(\mathbf {a} \cdot \nabla )\mathbf {b} .
\]
Hence,
\[
  \nabla \times (\bfu \times \vort ) =
  \bfu \ (\nabla \cdot \vort )
  -\vort \ (\nabla \cdot \bfu )
  +(\vort \cdot \nabla )\bfu
  -(\bfu \cdot \nabla )\vort.
\]
The first term is always zero, since the vorticity is divergence-free
(it being the curl of a field). The second is also zero,
since we have already assumed constant density (which, according
to the discussion in \S \ref{sec:incompressibility},
implies the velocity is divergence-free).
In this case,
\begin{equation}
  \label{eq:conv_accel_as_vort}
  \nabla \times (\bfu \times \vort ) =
  (\vort \cdot \nabla )\bfu
  -(\bfu \cdot \nabla )\vort,
\end{equation}
and we may write
\[
  \frac{\partial \vort }{\partial t}
  + ( \bfu \cdot \nabla )\vort =  (\vort \cdot \nabla )\bfu .
\]
The left hand side is the advective derivative. Then:
\[
  \frac{d \vort }{d t}  =  (\vort \cdot \nabla )\bfu ,
\]
which means vorticity is ``almost'' conserved. The righ-hand side
represents an advection of velocity by vorticity, which sounds a bit
upside-down. This term may be neglected if the flow is slow (since it
involves a term quadratic in the velocity), to arrive at
$\frac{d \vort }{d t}\approx 0$. We will see, in Section
\ref{sec:NS_vort}, how viscosity adds a diffusion term to this
equation.


In any case, the equation involves only the velocity and its curl, the
pressure having been ``curled-away.'' Indeed, there is a class of
numerical methods, called ``vortex methods'' in which this
simplification is employed.  In many applications, however, its
usefulness is limited. For example, boundary conditions may be
difficult to define for these terms.

There is, however, an important consequence of this equation: if a
given flow is curl-free at some instant, it must \emph{remain} so
at every other time (both future, and past). This is because the equation
is first order in time, and a null right-hand side translates into
a null change. We will see later that when viscosity is introduced,
a diffusion term appears which is second order in space. This term
is, in many cases, responsible for the generation of vorticity.
In the clearest instance, the no-slip boundary condition close
to solid walls, by which the velocity must be zero there, creates
zones of vorticity generation. That boundary condition cannot be
enforced within the Euler framework, because it only features first order
spatial derivatives of the velocity.
