\section{Vorticity}

The momentum equation \ref{eq:Euler_momentum} is still quite daunting
due to its nonlinear term in the convective derivative.

In order to make progress, it was Lamb's idea in 1895 to use the
following identity for the vector product of any vector field and its
curl:
\[
\bfu \times ( \nabla\times \bfu) =
\frac12 \nabla u^2 - \bfu (\nabla \bfu )
\]
(in fact, the identity is somewhat more general, see exercise
\ref{ex:vector_identity}).

Introducing the name ``vorticity'' for the curl of the velocity field,
\[
\vort = \nabla\times\bfu ,
\]
we may write the momentum equation as
\begin{equation}
  \label{eq:Euler_momentum_w_vort}
  \frac{\partial \bfu }{\partial t} +
  \frac12 \nabla u^2 - \bfu \times\vort =
  - \frac{1}{\rho} \nabla p 
  + \bfg .
\end{equation}

Now, recalling that the curl of a gradient is always zero,
we may apply the curl operator to the whole equation, to get
\[
\frac{\partial \vort }{\partial t} =
\nabla\times(\bfu\times\vort) .
\]
This is a very remarkable equation, involving only the velocity
and its curl. Indeed, there is a class of numerical methods,
called ``vortex methods'' in which this simplification is employed.
In many applications, however, its usefulness is limited. For
example, boundary conditions may be difficult to define for these terms.

There is, however, an important consequence of this equation: if a
given flow is curl-free at some instant, it must \emph{remain} so
at every other time (both future, and past). This is because the equation
is first order in time, and a null right-hand side translates into
a null change. We will see later that when viscosity is introduced,
a diffusion term appears which is second order in space. This term
is, in many cases, responsible for the generation of vorticity.
In the clearest instance, the no-slip boundary condition close
to solid walls, by which the velocity must be zero there, creates
zones of vorticity generation. That boundary condition cannot be
enforced within the Euler framework, because it only features first order
spatial derivatives of the velocity.
