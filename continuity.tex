\section{The concept of particle}

\section{The continuity equation}
\label{sec:continuity}

\begin{equation}
  \label{eq:continuity}
  \frac{\partial \rho }{\partial t} +  \nabla \cdot (\rho \bfu ) = 0 
\end{equation}

\subsection{Incompressibility}

By ``incompressible'' it is usually meant that variations in $\rho$
(spatial or temporal) are negligible. Continuity then readily implies:
\[
\nabla \cdot  \bfu  = 0 ,
\]
I.e., the velocity field must be divergence-free --- another adjective
is ``solenoidal'', as an analogy with the magnetic field. The latter,
as no magnetic charges (``monopoles'') have ever been seen, satisfies
this equation.

A technical issue is incompressibility is often defined as $\nabla
\cdot \bfu = 0 $. This does \emph{not} imply that $\rho$ is constant,
even though its reverse is true. What does imply is that $\rho$ is
constant at every particle, or along streamlines, as will be discussed
in \ref{sec:continuity2}.
