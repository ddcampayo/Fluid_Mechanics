\section{The concept of small regions}

In these notes, we will be dealing with regions that are ''infinitesimally small''
quite a lot. This is because fluid dynamics are part of what is called
continuum mechanics, in which bodies are supposed to be continuous in
a mathematical sense. By this me mean, we can pieces of any material
as small as we want, so that we may use the mathematical language of calculus.
Nevertheless, these regions are large enough so that may be considered
continuous, and have defined macroscopic physical parameters: density, mass,
volume, temperature, and so on.

This is an excellent approximation, most of the time, since the number of molecules
involved in a flow is usually so large, that even for a very small region of it, there
are very many molecules. An interesting exercise is to go through an order-of-magnitude
calculation of these numbers.

In these notes, we will consider two sort of regions: volume parcels\index{parcel}, which are very
small regions that are fixed in space, and particles\index{particle}, which
are small regions of the moving fluid (thus, in general, changing their positions with time).
A parcel is related to an ``Eulerian'' point of view, in which a flow moves over
fixed coordinates --- particles are related to the complementary, ``Lagrangian,'' view of the particle as moving with the flow, as will be discussed in \ref{sec:continuity2}.



\section{The continuity equation}
\label{sec:continuity}

Let us consider a fixed that is cubic in shape, with one of their
corners at the origin and the three incident edges aligned with the
Cartesian coordinates. This choice is completely general, given our
freedom in defining a region (as long as we make it infinitesimally
small at the end), and in origin and axes for coordinates.

There may be a mass influx though the left face of the parcel due
to advection (i.e. the existence of a velocity field). For a small
time $dt$, it will be given by
\[
\left. d m \right|_\text{l} =
\rho ( u_x \, dt) \, dy\, dz .
\]
Indeed, all the fluid material that a distance $\delta= u_x \, dt$
away, or closer, will pass through the surface. The whole mass influx
is therefore $\rho$ times $\delta dA$, where $dA=dy dz$ is the cross-
sectional area. (Since the parcel is cubic, $dx=dy=dz$, but it is
clearer to use different symbols.)

At the right wall, the mass change will be negative (if $v_x$ is positive
there):
\[
\left. d m \right|_\text{r} =
 -  \rho' ( u'_x \, dt) \, dy\, dz .
\]

The mass change due to the horizontal component of the velocity field is
therefore
\[
\left. d m \right|_x =
  \left[  \rho'  u'_x -  \rho  u_x \right] \, dt \, dy\, dz .
\]

Notice that we have allowed for the possibility that the combination
$\rho u_x$ may be somewhat different at the left and right
walls. Indeed, if they were equal, there would be no net mass change,
since the mass that enters would equal that leaving. Nevertheless,
since $dx$ is small, we may approximate the value at the right
wall in a Taylor series:
\[
\rho'  u'_x \approx  \rho  u_x  +
\frac{\partial  \rho  u_x}{\partial x} dx .
\]
Therefore,
\[
\left. d m \right|_x \approx
-\frac{\partial  \rho  u_x}{\partial x} \, dt \, dx \, dy\, dz .
\]
Or,
\[
\frac1V \left. \frac{d m}{d t} \right|_x \approx
-\frac{\partial  \rho  u_x}{\partial x}
\]

Since $dm/V=\rho$, the density of the parcel, and $V$ does not change with time, since the parcel does not move, we may write, in the
limit of $dt\to 0$,
\[
\left. \frac{\partial \rho}{\partial t} \right|_x =
-\frac{\partial  \rho  u_x}{\partial x} .
\]

Notice the usage of the $\partial$ symbol: since the parcel is
fixed, with matter entering and leaving, this is the correct
symbol.

In order to get the total rate of change in density, we have to
add up the other two directions. In doing so, we obtain
\begin{equation}
  \label{eq:continuity0}
  \frac{\partial \rho}{\partial t} =
  -\frac{\partial  \rho  u_x}{\partial x}
    -\frac{\partial  \rho  u_y}{\partial y}
    -\frac{\partial  \rho  u_z}{\partial z}  =
    -\nabla\cdot ( \rho  \bfu) .
\end{equation}
In the last equation the divergence of field $ \rho \bfu$ is
seen to appear naturally.

This is the correct expression, which is usually written as
\begin{equation}
  \label{eq:continuity}
  \frac{\partial \rho }{\partial t} +  \nabla \cdot (\rho \bfu ) = 0 .
\end{equation}


A more direct procedure to get it involves integrals, rather
than differentials.

By definition, mass is the integral of the density:
\[
m = \int_V \rho d\bfr \quad \Rightarrow \quad \frac{\partial m}{\partial t} = \int_V \frac{\partial \rho}{\partial t}  d\bfr,
\]
where the latter time derivative may be taken since $V$ is a fixed parcel, that does not change with time.

A net mass flux $\rho\bfu$  entering or leaving parcel $V$ may cause changes in it:
\[
\frac{\partial m}{\partial t} =
- \oint_{\partial V} \rho \bfu \cdot \bfn dS ,
\]
where $\partial V$ is the surface enclosing volume $V$. The minus sign
appears because the surface normal, $\bfn$ points from the inside
toward the outside of the surface. We may transform the latter
integral into a volume integral from Gauss' theorem:
\[
\frac{\partial m}{\partial t} =
- \int_V \nabla\cdot (\rho \bfu ) dV .
\]
Equating both mass change rates,
\[
\int_V \rho d\bfr = - \int_V \nabla\cdot (\rho \bfu ) dV .
\]
Since the identity is valid for any volume $V$, the integrands may
be equal. This fact leads to the same expression as before.

In the derivations above, no mention has been made to the nature of
the $\rho$ field. Therefore, the results are general for any field $A$,
corresponding to a physical quantity that is conserved while carried
by a velocity field. In general,
the resulting equation is called the convection\index{convection}, or advection\index{advection},
equation:
\begin{equation}
	\label{eq:convection}
	\frac{\partial A }{\partial t} +  \nabla \cdot (A \bfu ) = 0 .
\end{equation}



\subsection{Incompressibility}
\label{sec:incompressibility}

By ``incompressibility''\index{incompressibility}, it could be meant
that variations in $\rho$ (spatial or temporal) are
negligible. Continuity then readily implies:
\[
\nabla \cdot  \bfu  = 0 ,
\]
I.e., the velocity field must be divergence-free --- another adjective
is ``solenoidal'', as an analogy with the magnetic field. The latter,
as no magnetic charges (``monopoles'') have ever been seen, satisfies
this equation.

Indeed, this is sometimes the meaning of the term.  A technical issue
is incompressibility is often defined, rather, as
$\nabla \cdot \bfu = 0 $. This does \emph{not} imply that $\rho$ is
constant, even though its reverse is true. Indeed, in this case:
\[
\frac{\partial \rho }{\partial t} +  \nabla \cdot (\rho \bfu ) =
\frac{\partial \rho }{\partial t} +   \bfu \cdot \nabla \rho
+ \rho \underbrace{\nabla \cdot \bfu}_{=0} = 0
\]

\begin{equation}\label{eq:continuity_incomp}
	\frac{\partial \rho }{\partial t} +   \bfu \cdot \nabla \rho = 0 .
\end{equation}

As will be discussed in Section \ref{sec:continuity2}, this means that the
total (or ``convective'') time derivative of $\rho$ is zero. This
implies that $\rho$ is constant at every moving particle, or along
streamlines. Again, this would apply to any field $A$ of a conserved
quantity.

In order to make the distiction clear, the term ``divergence-free'' (or
``divergence-less'', or ``solenoidal'') is sometimes used.\index{divergence-free}
