\section{Monolayers and membranes}

In their influential 1975 article Saffman and Delbr{\"u}ck
\cite{saffman1975brownian} discussed several ways out of this paradox,
and found out the most satisfactory way was to take into account the
viscosity of the surrounding liquid, $ \mu_\mathbf{f}$.

In \cite{lubensky1996hydrodynamics}, Lubensky and Goldstein proposed a
modification of the original equation for 2D liquids. This are indeed
feasible in a lab, as Langmuir monolayers \index{Langmuir monolayer},
in which a one molecule thick layer of lipids forms between water and
air. Also, in membranes, where a 2D bilayer lipid is surrounded on
both sides by water (these are more involved to study, and may be
prone to curve). The equation of motion, in the creeping regime is
\begin{equation}
  \label{eq:Stokes_2D}
  \mu \nabla^2 \bfu - \nabla p = -\bff(\bfr) -\bff'(\bfr) ,
\end{equation}
%
where all variables are two-dimensional: $\mu$ is the 2D shear
viscosity, $p$ the 2D pressure, and $\bff$ a force per area. All
variables have their physical dimensions changed from the 3D case, but
for the position $\bfr$ and the velocity $\bfu$ --- both are in-plane,
but their units do not change: length and length per time.

In addition, the 2D flow is incompressible:
\[
  \nabla\cdot \bfu = 0
\]

The solvent produces an additional term $\bff'(\bfr) $, to be added to
any external force that causes the flow (which may be an external
forcing due to the motion of an immersed body, or perhaps due to a
concentration gradient \cite{duque2018}. Of course, this surrounding
flow is caused by the 2D fluid, so this will be a sort of
self-interaction effect, as we will see.

The solvent itself satisfies another momentum equation. It the solvent
is taken to be also in the creeping regime:
\[
  \mu' \nabla'^2 \bfu' -\nabla' p' = 0 .
\]
In this section, primes refer to volume properties of the solvent
(also for $\nabla'$, which is our usual, 3D, del operator). In this
equation $\mu'$ is the ordinary solvent shear viscosity, often simply
water.  The solvent is also incompressible:
\[
  \nabla'\cdot \bfu' = 0 .
\]

The reader may wonder how the 2D fluid affects the 3D one which is
underneath (this is the case of monolayers, while in membranes there
are two such regions, above and below.) The answer is, through the
boundary condition at the surface:
\[
  \bfu'(z=0) = \bfu .
\]
In addition to this condition, another one must be set. Usually,
disturbances die out quite fast (as we will see below), so the
deep-water limit is a good approximation:
\[
  \bfu'(z \to -\infty ) = 0 .
\]
A more general condition $\bfu'(z = -H ) = 0$ is also easy, but
somewhat obscures the discussion.

Let us consider the 2D Fourier transform of our in-plane velocity
field, $\bfu_\bfq$. The wave-vector $\bfq$ is also a 2D vector. Let us
consider this ansatz for the solvent velocity field:
\[
  \bfu' =  \frac 1{(2\pi)^2}  \int d\bfq \, \bfu_\bfq e^{ i\bfq\cdot\bfr} e^{q z} .
\]
Notice that this is a half-Fourier inverse transform, only in $q_x$
and $q_y$, but not on $q_z$. What we have here is a dilution of each
of the 2D Fourier components $\bfu_\bfq$ as we go deeper into the
solvent, each decaying as $e^{q z} $ (remember, $z$ is
negative!). This means the sharper features will wash out first, and
the longest-wavelength features will be the ones that will extend
deeper.

Notice that this velocity field is laminar: it is always parallel to
the 2D layer (i.e. it has no $z$ component). It automatically
satisfies incompressibility. It is also harmonic: $\nabla^2 \bfu' = 0$
(for this, it is imperative that the decay is precisely
$e^{qz}$. These two facts imply that the pressure field is not neede,
and we may take it to be constant, or zero.

Lastly, the force this velocity field will exert on the planar
interface can only be a wall shear stress. The only non-zero value is
\[
  \bff' = \mu' \left. \frac{\partial \bfu'}{\partial z} \right_{z=0} .
\]

With our previous ansatz,
\[
  \bff' = \frac 1{(2\pi)^2} \mu' \int d\bfq \, q \bfu_\bfq e^{
    i\bfq\cdot\bfr} .
\]

This is an interesting result. It is clear that the 2D Fourier
transform of the solvent force is
\[
  \bff'_\bfq =  \mu' q \bfu_\bfq .
\]

Therefore, if we cast our original 2D momentum equation
\ref{eq:Stokes_2D} into Fourier space:
\[
-  \mu q^2 \bfu_\bfq - i \bfq p_\bfq = -\bff_\bfq -\bff'_\bfq ,
\]
we have
\[
-  ( \mu q^2 +  \mu' q ) \bfu_\bfq - i \bfq p_\bfq = -\bff_\bfq .
\]

We may now repeat the steps by which we obtained
\ref{eq:u_sol_Fourier}, with the result
\begin{equation*}
\bfu_i(\bfq) =  \sum_j T_{ij} \bff_j(\bfq),
\end{equation*}
%
but now the Oseen tensor is re-defined:
\[
T_{ij}(\bfq) := \frac 1{\mu q^2 +  \mu' q  } \left[
  \delta_{ij} - \frac{ \bfq_i  \bfq_j}{q^2} 
\right].
\]

The denominator is usually written as
\[
  \mu q^2 +  \mu' q =:
  \mu (q^2 +  q \xi ) ,
\]
where $\xi:= \mu'/\mu$ is the Saffman and Delbr{\"u}ck length
\index{Saffman and Delbr{\"u}ck length }. (Remember the two
viscosities have different dimensions, so their ratio is indeed a
length). This length separates two regimes.  One of them is the
high-$q$, short wave-length regime, when $q \gg 1/\xi$. In this case
the effect of the solvent is negligible, as we recover our previous 2D
result. The other is the low-$q$, high wave-length regime, when
$q \ll 1/\xi$. In this case the $q\xi$ term dominates, and the tensor
switches over to a $1/q$ increase which makes it Fourier
invertible. Stokes' paradox is therefore removed! Sadly, the resulting
expressions cannot be annalitically expressed in general (they can in
those two limits.)

For membranes the discussion is exactly the same, with two mirroring
fluid zones above and below the membrane, both of them affecting the
membrane. The only difference is therefore that in this case
\[
T_{ij}(\bfq) = \frac 1{\mu q^2 +  2 \mu' q  } \left[
  \delta_{ij} - \frac{ \bfq_i  \bfq_j}{q^2} 
\right],
\]
and the Saffman and Delbr{\"u}ck length $\xi = 2 \mu'/\mu$.

