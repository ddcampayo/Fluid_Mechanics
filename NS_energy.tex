In addition to continuity and momentum, there is an additional
Navier-Stokes equation for the energy.

It contains the previous expression for ideal fluid, plus a term
expressing energy dissipation by viscosity.

We must re-evaluate the work done on each of the faces of the
particles due to stresses. For example, on the left wall the energy
due to stress forces is
\[
dW(\mathrm{left})  =
 -(u_x dt)   \tau_{xx} dy dz
 -(u_y dt)   \tau_{xy} dy dz
 -(u_z dt)   \tau_{xz} dy dz .
\]

Each of the stresses on this wall does work only in its direction of
motion: $\tau_{xx}$ is compression and will feature a $-p$ term, which
$\tau_{xy}$ and $\tau_{xz}$ produce shear forces. The minus sign stem
from the sign convention, since on this wall shear stresses have
directions opposed to the Cartesian axes. Similarly,
\[
dW(\mathrm{right})  =
 (u_x' dt)   \tau'_{xx}  dy dz
+(u_y' dt)   \tau'_{xy} dy dz
+(u_z' dt)   \tau'_{xz} dy dz ,
\]
where the primed values mean those fields may be different from the
left wall. Expanding in Fourier series, and adding everything up,
\[
dW(\mathrm{left,right})  =
 \frac{\partial u_x \tau_{xx}}{\partial x}  dt dx  dy dz  +
 \frac{\partial u_y \tau_{xy}}{\partial x}  dt dx  dy dz  +
 \frac{\partial u_z \tau_{xz}}{\partial x}  dt dx  dy dz ,
 \]
or, we find for the power
\[
\frac{dW(\mathrm{left,right})}{dt}  =
dV 
\frac{\partial }{\partial x}
\left(
 u_x \tau_{xx} +
 u_y \tau_{xy} +
 u_z \tau_{xz}
 \right) =
dV 
\frac{\partial }{\partial x}
\sum_j u_j \tau_{xj}
\]

Adding the other four walls, we have:
\[
\frac{dW}{dt}  =
dV
 \sum_i \frac{\partial }{\partial x_i} \sum_j u_j \tau_{ij} .
\]

Since the stress tensor is symmetric, we may write the latter as
\[
\frac{dW}{dt}  =
dV
\nabla\cdot ( \tau \cdot  \bfu ) ,
\]

Now, the energy equation is, from the First Law:
\[
\frac{dE}{dt}  = \frac{dW}{dt} + \frac{dQ}{dt} =
dV \nabla\cdot ( \tau \cdot  \bfu ) - dV\nabla\cdot\bfq .
\]
(The second term, due to heat flux, does not change from the inviscid
case.)

Dividing by the mass of the particle,
\begin{equation}
\label{eq:NS_en_2}
\rho \frac{d \epsilon }{dt}  = 
 \nabla\cdot ( \tau \cdot  \bfu ) - \nabla\cdot\bfq .
\end{equation}
(In this equation, $\epsilon=E/M$, as in \ref{}, not the strain rate.)

The term $\nabla\cdot ( \tau \cdot \bfu )$ may be written applying the
chain rule carefully:
\begin{equation}
  \nabla\cdot ( \tau \cdot  \bfu ) =
  \tau : \nabla\bfu + \bfu\cdot(\nabla\cdot\tau) ,
\label{eq:nabla_tau_u}
\end{equation}
where ``$:$'' means a total reduction of two tensors, $a:b=\sum_{ij}
a_{ij} b_{ij} $, and $\nabla\bfu$ is the tensor with components
$\partial u_i/\partial x_j$ (as introduced in the Euler
equation \ref{eq:Euler_2}).

We now just follow the steps already employed when deriving the energy
equation for an inviscid fluid (Eqs \ref{eq:}).  The $\nabla\cdot\tau$
appears in the general Navier-Stokes momentum equation \ref{eq:NS}:
\[
 \nabla \cdot \tau =
\rho \left( \frac{d\bfu }{dt}    - \bfg \right) .
\]

Therefore,
\[
\bfu\cdot(\nabla\cdot\tau) =
 \rho \left[
 \frac12 
  \frac{d u^2 }{dt}    - \bfg\cdot\bfu
  \right] =
   \rho
  \frac{d (u^2 /2 - \bfg\cdot\bfr ) }{dt}    .
\]

The conclusion is then that the energy equation \ref{eq:NS_en_2} may be
expressed as a law for the specific internal energy:
\[
\rho \frac{d e }{dt}  = 
  \tau : \nabla\bfu - \nabla\cdot\bfq .
\]
This looks more similar to \ref{eq:} for an ideal fluid if we split the
stress tensor into the pressure diagonal and the rest:
\[
\tau = \tau' - p \eye \qquad \implies \qquad
 \tau' : \nabla\bfu = \tau' : \nabla\bfu - p \divu ,
\]
hence
\[
\rho \frac{d e }{dt}  =  - p \divu - \nabla\cdot\bfq  + \Phi .
\]

The term $\Phi$ collects the result of viscosity and is termed
the ``dissipation function'':
\[
\Phi = \tau' : \nabla\bfu .
\]

This term should always be positive if the Second Law is to hold:
viscosity can only subtract energy from the system, never add it.
Up until now our derivation has been generic. For a Newtonian fluid,
however, one may go a bit further, and write:
\begin{equation}
  \Phi = \mu
  \left[
    % 2 \sum_i \left( \frac{\partial u_i}{\partial x_i}\right)^2
    % \sum_i \left( \frac{\partial u_i}{\partial x_i}\right)^2
    2 \sum_i \epsilon_{ii}^2 +
    4  (\epsilon_{12}^2 + \epsilon_{13}^2 + \epsilon_{23}^2 )
  \right]+
  \lambda \divu^2 .\label{eq:Phi_Newtonian}
\end{equation}

This looks deceptively positive unconditionally. However, there is no
reason $\lambda$ should be positive. It is a simple exercise to show
that the conditions this term be positive are:
\begin{equation}
  \mu \ge 0 \qquad 3\lambda + 2\mu \ge 0 .\label{eq:mu_lambda_cond}
\end{equation}

TODO: exercise on this

The first one comes as a relief, since a $\mu$ would be quite
unphysical. The second limits the value of $\lambda$ to regions equal
to, or above, $ - (2/3) \mu$. This latter term is precisely zero for
``Stokesian'' fluids, as is obvious (since the corresponding term does
not appear at all in the stress tensor for these hypothetical fluids.)

\section{Exercises}

\begin{enumerate}
\item Check identity \ref{eq:nabla_tau_u}. Hint: use element notation.
\item Show that the conditions in \ref{eq:mu_lambda_cond} are indeed
  needed in order $\Phi$ in \ref{eq:Phi_Newtonian} be always
  positive. (Hint: look for ``postitive-definite quadratic form''. The
  expression for $\Phi$ can be expressed in such a way, and three
  conditions are obtained for this positiveness. However, one of them
  is $2\mu-\lambda \ge 0$, which is less restrictive than the other
  two taken together, so only the two conditions quoted remain.)
\end{enumerate}

