\section{Poiseuille flow}

\subsection{Planar flow}

As with Couette flow, let us assume the only component
of the velocity field is $u_x(y)$, a function of $y$ only.

The steady 2D Navier-Stokes equations read
\begin{align}
  0 & =   - \frac{\partial p}{\partial x} +
  \nu
  \frac{\partial^2 u_x}{\partial y^2}
   \\
  0 &=   - \frac{\partial p}{\partial y} .
\end{align}
where $p$ is actually $p/\rho$. The second one stablishes that $p$ is
a function of $x$ only. But in the first one, its derivative is
related to a second derivative of a funcion of $y$ only. It follows
that both equations should equal some constant:
\[
\frac{\partial p}{\partial x} =
 \nu
  \frac{\partial^2 u_x}{\partial y^2} = -c .
\]
I.e. $p= - c x$, plus a constant pressure which makes no
difference. Notice the minus sign: we consider a pressure drop in
the $x$ direction if $c > 0$.

For the velocity, we must solve for
\[
  \frac{\partial^2 u_x}{\partial y^2} = -\frac{c}{\nu} ,
\]
given the boundary conditions $u_x(y=0)=u_x(y=L)=0$.

The solution is
\begin{equation}
  \label{eq:Poiseuille_u}
  u_x=\frac{c}{2\nu} y (L-y)  
\end{equation}

The flow is
\[
Q= \frac{HL c L^2}{12\nu} ,
\]
and the mean velocity,
\[
\bar{u}= \frac{c L^2}{12\nu} ,
\]

Which let us write, more elegantly,
\[
u_x=6 \bar{u} \,  \frac{y}{L} \left( 1- \frac{y}{L}\right). 
\]


\subsection{Temperature}

As in Couette planar flow, \ref{sec:}, the temperature equation
reduces to
\[
0 =  k \frac{\partial^2 T}{\partial y^2} +
\mu  \left( \frac{\partial u_y}{\partial x} \right)^2,
\]
or
\[
0 =  k \frac{\partial^2 T}{\partial y^2} +
36 \mu \bar{u}^2 \left[
  \frac{1}{L} \left( 1- 2 \frac{y}{L}  \right)
  \right]^2 .
\]

Casting it into dimensionless form,
\[
0 =  k \frac{T_0}{ L^2} \frac{\partial^2 T^*}{\partial y^{*2}} +
36 \frac{\mu \bar{u}^2}{L^2}  \left( 1- 2 y^*  \right)^2 ,
\]
or
\[
 \frac{\partial^2 T^*}{\partial y^{*2}} = 
  - 36 \mathrm{Br} \left( 1- 2 y^*  \right)^2 ,
\]
where the  Brinkman number is again as in Eq. \ref{eq:}
\[
\mathrm{Br} =
\frac{ \mu \bar{u}^2 }{ \kappa T_0 } .
\]

If we define $s = 2  y^* - 1$, then
\[
 4 \frac{\partial^2 T^*}{\partial s^{2}} = 
 - 36 \mathrm{Br} s^2,
 \]
 or
\[
 \frac{\partial^2 T^*}{\partial s^{2}} = 
 - 9 \mathrm{Br} s^2.
\]


The final solution is then:
\[
T^* = 1 + \frac{T_1-T_0}{T_0} y^* +
\frac34 \mathrm{Br}  \left( 1 - s^{4} \right) .
\]
The second term is seen to vanish at the two walls (where $s=\pm 1$),
while providing the correct second derivative. In terms of $y^*$,
\[
T^* = 1 + \frac{T_1-T_0}{T_0} y^* +
\frac34 \mathrm{Br}  \left[ 1 -  \left( 2  y^* - 1 \right)^{4} \right] .
\]




\subsection{Flow in circular pipes}

The solution is
\[
u_x=\frac{c}{4\nu} \left( R^2 - r^2 \right) .
\]


The flow is
\[
Q= \frac{c \pi R^4}{8\nu} ,
\]
and the mean velocity,
\[
\bar{u}= \frac{c R^2}{8\nu} ,
\]



\subsection{Temperature}


At variance with planar flows, \ref{sec:}, the temperature equation
must be written in polar coordinates:
\[
0 =
k
\frac{1}{r} \frac{d}{dr} \left[
  r \frac{dT}{dr} \right] +
\mu  \left( \frac{d u_z}{d r} \right)^2,
\]
or
\[
0 = k
\frac{1}{r} \frac{d}{dr}
\left[
  r \frac{dT}{dr}
\right] +
16 \mu  \bar{u}^2 \frac{r^2}{R^4}
\]

Casting it into dimensionless form,
\[
0 = k T_\mathrm{w}
\frac{1}{R^2}
\frac{1}{r^*} \frac{d}{dr^*}
\left[
  r^* \frac{dT^*}{dr^*}
\right] +
16 \mu  \bar{u}^2 \frac{1}{R^1} r^{*2},
\]
or
\[
\frac{1}{r^*} \frac{d}{dr^*}
\left[
  r^* \frac{dT^*}{dr^*}
\right] =
- 16 \mathrm{Br}  r^{*2},
\]
where the  Brinkman number is as in Eq. \ref{eq:}.
In order to solve it, we change it to
\[
 \frac{d}{dr^*}
\left[
  r^* \frac{dT^*}{dr^*}
\right] =
 - 16 \mathrm{Br}  r^{*3},
\]
or
\[
 \frac{dT^*}{dr^*} =
 - 4 \mathrm{Br}  r^{*3} + \frac{c}{r^*} .
 \]
 Then,
\[
T^* = - \mathrm{Br}  r^{*4} + c \log( r^* ) + d .
 \]

 The $log$ term has a singularity at $r^*=0$, so $c=0$. The other
 constant has to be fixed in order to comply with the boundary
 condition $T^*(r^*=1)=1$. The final answer is
\[
T^* =  1 + \mathrm{Br} \left( 1 - r^{*4} \right) ,
\]
or, bringing back the units for length and temperature:
\[
T =  T_\mathrm{w}  + \mathrm{Br} \left[ 1 - \left(\frac{r}{R}\right)^4 \right] ,
\]
