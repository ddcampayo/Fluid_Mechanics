While certainly not as involved as dynamics, the static case of the
equations of fluid mechanics is nevertheless interesting. This secion
applies even to real fluids (with viscosity), because stresses are
constant on them (otherwise, a viscous force would arise that would
cause a motion.)

In equilibrium, all time derivatives (both partial, and material)
are zero, and the Euler momentum Equation \ref{eq:Euler_momentum}
reduces to
\begin{equation}
  \label{eq:hidrostatics}
  \nabla p = \rho \bfg .
\end{equation}


\section{Incompressible fluids}

The simplest case is when the density is constant. In this case,
\[
\nabla p = \rho \bfg \qquad\implies\qquad
p= p_0 + \rho \bfg \cdot \bfr ,
\]
Since,
\[
\nabla (\bfg \cdot \bfr) =
\nabla ( - g z ) = -g \bhe_z = \bfg ,
\]
where we assume the acceleration of gravity points toward negative
values of the $z$ coordinate: $\bfg = -g \bhe_z$.

Of course, the expression for the pressure in terms of $\bfg \cdot
\bfr $ is general and does not depend on our choice of Cartesian
axes. Nevertheless, it is usually written in function of the
``depth'', this being $h=-z$ in our choice (in general, the depth is
the coordinate at which we go against gravity). This means,
\[
p= p_0 + \rho g h.
\]

This means that pressure should increase as we go deeper. This can be
deduced from much simpler grounds, computing the pressure difference
between two parallel sides of a prismatic object immersed in a
fluid. This pressure difference translates into a net vertical force
that must balance the force due to gravity. The same conclusion is
reached. This is the standard way to derive the law of floating bodies
(Archimedes' law), also for the case of partly-floating boadies
(icebergs, boats\ldots).

The pressure features a reference pressure which should be provided in
order to fix the absolute value of the pressure. This reference is in
principle not important. Indeed, in most application ``pressure''
actually refers to excess pressure over atmospheric pressure, i.e.
manometric pressure. \index{manometric pressure} Notice also that it
is only gradients of the pressure which are relevant, never the value
itself. Exceptions arise when thermodynamics is relevant --- e.g. a
pressure can never be negative.

A usual case in which the pressure is known is at a liquid free
surface, most commonly, the water-air surface. There, the pressure is
the atmospheric pressure (which depends on the weather), and the
liquid must have that same pressure (otherwise the interface is not in
mechanical equilibrium). This means that the pressure increases
linearly as we dive deeper into the water. The increase is given by
$\rho g h$. If we write $\rho g  =: p_0  / L $, the law becomes
\[
p= p_0 \left( 1 + \frac{h}{L} \right) ,
\]
so that the pressure increases two-fold at $h=L$, three-fold at
$h=2L$, etc. For water, this value is $L= p_0 / (\rho g h ) =
\SI{10.3}{\meter}$. This is a well-known result for scuba divers:
expect an increase of one atmosphere every $\SI{10}{\meter}$,
approximately.

Of course, this simple result is only true for constant density. In
actual fact, sea water has density variations, due not so much to
compression, or temperature (which is quite constant, at about
$\SI{4}{\celsius}$, at which the density is greater), but due to
salinity. Nevertheless, if we neglect those factors, we may get an
estimation of the greatest pressure attained in oceans of our planet.
At the end of the Marianas Trench, $h\approx\SI{11}{\kilo\meter}$, and
$p\approx 1060 p_0$, about $1000$ atmospheres. This is a good match to
the measured value of $1090 p_0$. A refined calculation involving the
compressibility of water shows that at this pressure water is about
$5\%$ denser than at standard conditions, a measurable change if still
small.



\section{Compressible fluids}

If compressibility is taken into account, there will be a relationship
between pressure and density at least (in a more realistic picture,
temperature also has to be included). We will limit our discussion to
one of the most common examples: ideal gases. The expremelly well know
relationship may be rewritten in order to include the mass density:
\begin{equation}
  \label{eq:ideal_gas_EOS}
  p V = n R T \qquad\implies\qquad p = \rho\frac{ R T}{m},
\end{equation}
where $m$ is the molar gas. In many cases, the temperature is not
constant, but we will assume it to be so for simplicity.

In this case,
\[
\nabla p = \rho \bfg \qquad\implies\qquad
\frac{\partial p}{\partial z} = -g \rho .
\]
Since gravity is parallel to the $z$ axis (in our convention), $p$ can
only vary in this direction. The opposite is interesting: a horizontal
variation in pressure can never reach equilibrium
\cite{landau2013fluid}.

With our equation of state,
\[
\frac{ R T}{m} \frac{\partial \rho}{\partial z} = -g \rho ,
\]
or
\[
\frac{\partial \rho}{\partial z} = -\frac{1}{L} \rho ,
\]
where the height scale is seen to be given by $L:=RT/(gm)$. Since
$RT/m= p_0/\rho_0$ at some known values of $p_0$ and $\rho_0$, we may
find it more convenient to write $p_0/(g \rho_0)$. Of course, our
known values are $p_0=\SI{1010}{\hecto\pascal}$ and
$\rho_0=\SI{1.22}{\kilo\gram\per\meter\cubed}$, standard air
conditions. The result is $L=\SI{8.47}{\kilo\meter}$. This tells us
that air has constant density and pressure at human scales (at a given
time, we know weather changes it), and variations are observed only at
high altitudes. The solution to the equation above is
\begin{align*}
  \rho &= \rho_0 e^{-x/L} \\
  p &= p_0 e^{-x/L}
\end{align*}

Even if our approximation of constant temperature is surely wrong, it
is not too wrong for the troposphere, where these variations are not
strong (remember this is absolute temperature, where a
$\SI{30}{\celsius}$ change is still around a $10\%$ variation.) We are
also ignoring other factors, such as air humidity.  In spite of this,
we can boldly estimate the pressure and density at the top of Mount
Everest to be about $\exp( - 8.848 / 8.47 ) \approx 35 \%$ lower than
at sea level (it is quite a coincidence that $L$ is so close to the
elevation of this mountain.) Indeed, typical readings indicate
pressures of about $\SI{400}{\pascal}$, with temperatures of
$\SI{-17}{\celsius}$ (not so very far below the standard value), and a
relative humidity of $40\%$. At the highest point at which our
approximation is valid, the tropopause,
$z\approx\SI{17}{\kilo\meter}$, and we find a reduction of about
$13\%$.

It is noteworthy that for climbers it is the \emph{oxygen} concentration
which is important. Since the oxygen molecule has a lower molar weight that
the air, the same calculation yields $L\approx\SI{14.5}{km}$, and a
O$_2$ concentration at the top of Mt. Everest of $\exp(-8/14.5) \approx 56 \% $
the value at sea level. This is in rough agreement with the value of $40\%$ that
is often quoted, the different being probably due to the temperature.

