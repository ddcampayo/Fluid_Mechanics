\section{Couette flow}
\label{sec:Couette}

\index{Couette flow}
As a simple solution, let us derive the flow that was given as an
example in our derivation of section \ref{sec:Newtonian}. A plane moves in the
$x$ direction, parallel of a fixed plane, and separated a distance $L$
from it. The velocity field is supposed to depend only on $y$ and have
reached a steady state. (Notice that these assumptions restrict our
solution space to a very limited choice. Since the equation are known
not to comply with unicity, there may be other solutions, as indeed
there are.)

While this particular geometry may seem artificial, the original
Couette aparatus used a fluid between two coaxial cylinders. It is
quite easy to assemble and is one of the first accurate
viscometers. This flat geometry may be thought of as the limit of a
thin fluid layer between the curved surfaces.

In this flow, the total derivative in the momentum equation is
zero. The partial derivative is zero in steady state, and the
non-linear term also is, since $\bfu\nabla \bfu$ is
\[
u_x \frac{\partial u_y}{\partial x} +
u_y \frac{\partial u_y}{\partial y} = 0 .
\]

Also, $\nabla\cdot\bfu=0$, so the flow will always be
incompressible. The pressure then is constant, since it does not need
to ``ensure'' incompressibility.

The equation reduces then to
\[
\mu \frac{\partial^2 u_x}{\partial y^2} = 0 .
\]

The viscosity is then seen to be of no importance (other than it is
needed to reach a steady state, as shown below). The solution is
simply a linear function of $y$. The particular shape of the function
is given by the boundary conditions. If we take the usual no-slip
conditions, the velocity matches the velocity at the walls:
\[
u_x(y=0) =  0 \qquad u_x(y=L) = u_0 .
\]
Therefore,
\[
u_x =   u_0 \frac{y}{L} .
\]
These are called Dirichlet boundary conditions, since they fix the
value of the field.

This is the solution assumed in previous sections, when introducing
the viscosity coefficient. Even if the latter does not appear in the
velocity field, the stress tensor has only one independent component:
\[
\epsilon_{xy}=
\epsilon_{yx}=
\mu \frac{u_0}{L} .
\]
The tensor is also constant throughout the fluid.

This has physical importance, since both plates will feel a total
stress force
\begin{equation}
  \label{eq:Couette_force}
  F= A \epsilon_{xy} = \mu A \frac{u_0}{L} .
\end{equation}


This force must be maintained on the moving plate in order to keep the
steady flow (the fixed one must be anchored, and should resist the
same force in order not to be dragged along). Energy must then be
provided to the system, which is dissipated by viscosity. The power
into the system will be
\[
F u_0 = \mu A \frac{u_0^2}{L} = \mu V  \left( \frac{u_0}{L}\right)^2 ,
\]
where $V=AL$ is the total fluid volume.

Lastly, let us consider the volumetric flux:
\[
Q = \int_A v_x = \int_0^H dz \int_0^L dy v_x(y) =
H  u_0 \int_0^L \frac{y}{L} dy =
H L  u_0 \int_0^1 y' dy' =
\frac12 A u_0 .
\]

The mean velocity is defined as $Q/A$. Therefore,
\[
\bar{u} = \frac12 u_0 ,
\]
so the solution may be written as
\[
u_x =   2\bar{u} \frac{y}{L} .
\]




\subsection{Start-up of Couette flow}

It is not too difficult to solve the Navier-Stokes equations for
non-steady Couette flow. In this case, the partial time derivative
will not be absent, and
\[
{\frac {\partial u_x}{\partial t}}=\nu {\frac {\partial
    ^{2}u_x}{\partial y^{2}}}.
\]

Boundary conditions are as above, and let us consider the initial
fluid is at rest.

It is easer to work out the solution by substrating the known steady state:
\[
u_x(y,t) = u(y,t)  + u_0 \frac{y}{L} \qquad
u(y,t) = u_x(y,t)  - u_0 \frac{y}{L} 
\]

The reason is that the boundary conditions for the velocity field
about the steady state are homegeneous:
\[
u(y=0,t) = u(y=L,t) = 0
\]

Then, one may use the standard technique of separation of variables:
\[
u(y,t) = Y(y) T(t) .
\]

The equation of motion then reads
\[
Y T' = \nu T Y'' \qquad\implies\qquad  \frac{T'}{ \nu T} = \frac{Y''}{Y} = -c
\]
The last equality follows because functions of two independent
variables can only be equal if constant.

Therefore
\[
Y''= - c Y ,
\]
whose $n$-th solution, given the boundary conditions, is
\[
Y_n= A_n \sin(n\pi y/L),
\]
where $n$ is an integer greater than zero. The constant must then be
$c=(n\pi/h)^2$. The corresponding time function is then,
\[
T_n=\exp(-c_n t) = \exp(-  n^2 \nu (\pi/L)^2   t) .
\]

In general, the solution will be a combination of all possible modes:
\[
u_x(y,t) =  u_0 \frac{y}{L}
+ \sum_n A_n
\exp(-  n^2 \nu (\pi/L)^2   t)  \sin(n\pi y/h),
\]
where it is clearly seen that mode $n$ decays in a time that has an
$n^2 /\nu $ dependence. The longest-lived one is then $n=1$, and modes
with shortest wavelength decay quadratically faster. Also, the
viscosity sets the time-scale of the process: high viscosity means
shorter relaxation times. In the limit of no viscosity all times
diverge, since a fluid without viscosity is unable to transmit the
stress produced by the moving plane.

The $A_n$ are given by the initial condition:
\[
u_x(y,t=0) =  
u_0 \frac{y}{L}
+ \sum_n A_n \sin(n\pi y/h) = 0 ,
\]
which is a standard exercise in Fourier series analysis. The result
may be shown to be
\[
u_x(y,t=0) / u_0  =  
 \frac{y}{L}
+\frac{2}{\pi}
 \sum_n \frac{(-1)^n}{n} \sin(n\pi y/h)  .
\]

This can also be written as 
\[
u_x(y,t=0) / u_0  =  
 \frac{y}{L}
-\frac{2}{\pi}
 \sum_n \frac{1}{n} \sin(n\pi (1-y/h))  ,
\]
where the last sine term is always starts with a positive slope close to $y=h$.
The negative sign of every term in the expansion means that all of them are
trying very hard in order to push down the final steady-state linear solution
toward the initial one, which is null.



\subsection{Temperature}


For the steady state, the temperature equation reduces to
\[
0 = k  \frac{\partial^2 T}{\partial y^2} + \Phi  .
\]
The dissipation function in this case is simply
\[
\Phi =   \mu  \left( \frac{\partial u_y}{\partial x} \right)^2 =
4 \mu \bar{u}^2  \left(\frac{1}{L}  \right)^2 .
\]

The equation for the energy is therefore
\[
0 = k \frac{\partial^2 T}{\partial y^2} +
4  \frac{ \mu \bar{u}^2  }{L^2} ,
\]
where $k$ is the thermal conductivity (units of power / (length
$\times$ temperature), the whole equation has units of power / volume
.) The boundary conditions needed may be the temperature at the two
walls:
\[
T(y=0) = T_0 \qquad
T(y=L) = T_1 ,
\]
also known as the ``no-jump'' temperature conditions. The fluid is
supposed to have the same temperature as the walls under this
framework. Others may be easily explored, such as fixed energy influx,
which translate into conditions for the temperature derivatives at the
walls (also known as Neumann boundary conditions). If the derivative
is null, one has an adiabatic wall (aka homegeneous Neumann).

Before solving the equation, let us cast it into dimensionless form,
by reducing the temperature by its value on one wall: $T^* =
T/T_0$. Similarly, $y^* = y/L$. Then,
\[
0 = k\frac{T_0}{ L^2} \frac{\partial^2 T^*}{\partial y^{*2}} +
4  \frac{ \mu \bar{u}^2  }{L^2} ,
\]
or
\[
 \frac{\partial^2 T^*}{\partial y^{*2}} = 
 -4  \frac{ \mu \bar{u}^2 }{ k T_0 }  =
 -4 \mathrm{Br} ,
\]
where we define the important Brinkman number: \index{Brinkman number}
\begin{equation}
  \label{eq:Brinkman}
  \mathrm{Br} =
  \frac{ \mu \bar{u}^2 }{ k T_0 } .
\end{equation}

The number measures the importance of viscous dissipation over
temperature dissipation.

The final solution is:
\[
T^* = 1 + \frac{T_1-T_0}{T_0} y^* +
2 \mathrm{Br}  y^* \left(1- y^* \right) .
\]
The first two terms ensure the boundary conditions are satisfied, and
would be the only ones pressent if there were no viscous
dissipation. The latter term provides the needed second derivative,
and vanishes at the walls.

