\section{Fourier space}

In Fourier space all the fields are given as \index{Fourier transform} of their Fourier components:
\begin{equation}
	\label{eq:F_q_to_r}
	\phi(\bfr) = \frac 1{(2\pi)^d}  \int d\bfq \, \phi(\bfq) e^{ i\bfq\cdot\bfr} .
\end{equation}
In some settings, this integral may be replaced by a sum over integer-labeled Fourier components
(and is therefore termed a Fourier series.)
Notice that in this section Fourier components are indicated by their arguments taking a wave-vector. This is not too elegant, and a tilde (``wiggly'')
hat should be used for them.

The Fourier components can be obtained by the direct Fourier transform:
\begin{equation}
	\label{eq:F_r_to_q}
	\phi_\bfq =                    \int d\bfr \, \phi(\bfr) e^{-i\bfq\cdot\bfr}  
\end{equation}

In the incompressible Navier-Stokes equation \eqref{eq:NS_usual}, the viscous and force terms are trivial in Fourier space, giving
\begin{equation*}
	\frac{\partial}{\partial_t} \mathbf{u} (\mathbf{k}) + \mathbf{a}(\mathbf{k}) =
	-[\nabla p ](\mathbf{k}) - \nu k^2 \mathbf{u}(\mathbf{k}) + \mathbf{f}(\mathbf{k}) ,
\end{equation*}
where $\mathbf{a}$ stands for the convective acceleration, $ \mathbf{a} := (\mathbf{u}\cdot\nabla)\mathbf{u} $. Ignoring, and the pressure, the Fourier modes are decoupled. For example, starting from a null velocity field, the solution is
readily obtained as
\begin{equation} \label{eq:Fourier_simple_solution}
	\mathbf{u} (\mathbf{k}, t ) =
	\frac{\mathbf{f}(\mathbf{k})}{\nu k^2}
	\left[ 1-  \exp \left( - \nu k^2 t  \right)   \right]  .
\end{equation}

This shows that the final velocity is given by $\mathbf{f}(\mathbf{k}) / \nu k^2 \sim  \mathbf{f}(\mathbf{k}) \lambda^2 / \nu $, so that longer wavelengths are more affected by a force.
Also, the typical relaxation time of a mode is given by $\tau= 1/( \nu k^2 ) \sim \lambda^2 / \nu$, so longer wavelengths take longer to reach equilibrium.



\subsection{Convective acceleration in Fourier space}

From its definition \(
\mathbf{a} := (\mathbf{u}\cdot\nabla)\mathbf{u} ,
\)
%
\[
\mathbf{a} = 
\left(
\sum_\mathbf{p} \mathbf{u}(\mathbf{p} ) e^{\mathi \mathbf{p}\cdot\mathbf{r}} 
\cdot\nabla
\right)
\sum_\mathbf{q} \mathbf{u}(\mathbf{q} ) e^{\mathi \mathbf{q}\cdot\mathbf{r}} .
\]

It is clearer perhaps in Fourier component form:
\[
a_i = 
\sum_\mathbf{p} u_j(\mathbf{p} )  e^{\mathi \mathbf{p}\cdot\mathbf{r}} 
\partial_j
\sum_\mathbf{q} u_i(\mathbf{q} ) e^{\mathi \mathbf{q}\cdot\mathbf{r}} ,
\]
where Einstein's summation convention is used. Also, $\mathi$ is used
for the imaginary unit, since the letter $i$ is used as an index for
Cartesian components. Then,
\[
a_i = \mathi
\sum_{ \mathbf{p}, \mathbf{q}}
u_j(\mathbf{p} )
q_j
u_i(\mathbf{q} )
e^{\mathi ( \mathbf{p}  +  \mathbf{q} ) \cdot\mathbf{r}} ,
\]
or
\[
\mathbf{a} = \mathi
\sum_{ \mathbf{p}, \mathbf{q}}
(\mathbf{u}(\mathbf{p} )
\cdot
\mathbf{q})
\mathbf{u}(\mathbf{q} )
e^{\mathi ( \mathbf{p}  +  \mathbf{q} ) \cdot\mathbf{r}} .
\]

The Fourier components will be given by \eqref{eq:F_q_to_r}, and by orthogonality of the
complex exponential functions:
\[
\mathbf{a}(\mathbf{k})= \mathi
\sum_{ \mathbf{p} + \mathbf{q} = \mathbf{k}}
(\mathbf{u}(\mathbf{p} )
\cdot
\mathbf{q})
\mathbf{u}(\mathbf{q} ) .
\]
This is a \emph{single} sum: for a given $\mathbf{k}$ we sum over all $\mathbf{p}$,
with $\mathbf{q}$ fixed by the sum rule $\mathbf{p} + \mathbf{q} = \mathbf{k}$.

Clearly, the convective acceleration is responsible for the coupling of the different Fourier modes.


Also, since $\mathbf{p} \cdot \mathbf{u}(\mathbf{p}) = 0$ by incompressibility, we
may change $\mathbf{q}$ to $\mathbf{q} + \mathbf{p} = \mathbf{k} $,
so
\[
\mathbf{a}(\mathbf{k})=\mathi
\sum_{ \mathbf{p} + \mathbf{q} = \mathbf{k}}
(\mathbf{u}(\mathbf{p} )
\cdot
\mathbf{k})
\mathbf{u}(\mathbf{q} ).
\]
The term $\mathbf{k}$ can be taken out of the summation, but it is awkward in vector form. Not so in components:
\[
a_i(\mathbf{k})=\mathi
k_j
\sum_{ \mathbf{p} + \mathbf{q} = \mathbf{k}}
u_j (\mathbf{p} )
u_i(\mathbf{q} ) .
\]



\subsection{Pressure in Fourier space}

Taking the divergence of the Navier-Stokes equation, \eqref{eq:NS_usual}, and applying incompressibility,
\begin{equation}\label{eq:Poisson}
	\nabla\cdot \mathbf{a} = -\nabla^2 p + \nabla\cdot \mathbf{f} ,
\end{equation}
which is the \index{Poisson pressure equation} (PPE).
In principle, this equation can be solved in real space, obtaining the pressure from it. Introducing its gradient back into \eqref{eq:NS_usual} the Navier-Stokes equation this is a possible way to solve the problem. (Indeed, this is the procedure in some computational algorithms.)

In Fourier space,
\[
-k^2 p (\mathbf{k} ) = - \mathi \mathbf{k} \cdot \mathbf{a}(\mathbf{k}) + \mathi \mathbf{k} \cdot \mathbf{f}(\mathbf{k}) ,
\]
or
\[
p (\mathbf{k} ) =  \frac{\mathi}{k^2} \mathbf{k} \cdot \mathbf{a}(\mathbf{k})
- \mathi \frac{\mathbf{k} \cdot \mathbf{f}(\mathbf{k})}{k^2}  .
\]


Hence,
\begin{align*}
	p (\mathbf{k} ) & = -\frac{1}{k^2} 
	\sum_{ \mathbf{p} + \mathbf{q} = \mathbf{k}}
	(\mathbf{u}(\mathbf{p} )
	\cdot
	\mathbf{k}
	)
	(
	\mathbf{u}(\mathbf{q} ) \cdot
	\mathbf{k}
	) 
	- \mathi
	\frac{\mathbf{k} \cdot \mathbf{f}(\mathbf{k}) }{k^2}     
	%\\
	%&= -\frac{1}{k^2} 
	%\left[
	%\sum_{ \mathbf{p} + \mathbf{q} = \mathbf{k}}
	%(\mathbf{u}(\mathbf{p} )
	%\cdot
	%\mathbf{k}
	%)
	%(
	%\mathbf{u}(\mathbf{q} ) \cdot
	%\mathbf{k}
	%) 
	%+ \mathi
	%(\mathbf{k} \cdot \mathbf{f}(\mathbf{k}) )
	%\right]
\end{align*}


Or, in components,
\[
p (\mathbf{k} ) = -\frac{k_j k_m}{k^2}
\sum_{ \mathbf{p} + \mathbf{q} = \mathbf{k}}
u_j(\mathbf{p} )
u_m(\mathbf{q} )
- \mathi
\frac{k_j f_j(\mathbf{k}) }{k^2} .
\]


The NS equation features the pressure gradient, which will be given in Fourier space as:
\[
[\nabla p ](\mathbf{k}) = \mathi \mathbf{k} p(\mathbf{k}) .
\]

I.e.
\[[\nabla p ](\mathbf{k}) =
-\mathi \frac{\mathbf{k} }{k^2} 
\sum_{ \mathbf{p} + \mathbf{q} = \mathbf{k}}
(\mathbf{u}(\mathbf{p} )
\cdot
\mathbf{k}
)
(
\mathbf{u}(\mathbf{q} ) \cdot
\mathbf{k}
) 
+ \mathbf{k}
\frac{\mathbf{k} \cdot \mathbf{f}(\mathbf{k})  }{k^2} 
\]




\subsection{Assembly of the final equation}


Recalling our previous expressions for the convective acceleration and pressure,
\begin{equation*}
	\partial_t \mathbf{u} (\mathbf{k})  =
	- \mathi  
	\sum_{ \mathbf{p} + \mathbf{q} = \mathbf{k}}
	(\mathbf{u}(\mathbf{p} ) \cdot \mathbf{k} )
	\left[ 
	1 - \mathbf{k} \frac{\mathbf{k} }{k^2} \cdot 
	\right]
	\mathbf{u}(\mathbf{q} )
	%
	- \nu k^2 \mathbf{u}(\mathbf{k}) +
	\left[ 
	1 - \mathbf{k} \frac{\mathbf{k} }{k^2} \cdot 
	\right]
	\mathbf{f}(\mathbf{k}) .
\end{equation*}

In component form,
\begin{equation*}
	\partial_t u_i (\mathbf{k})  =
	- \mathi  k_m
	\left[ 
	\delta_{ij} - \frac{k_i k_j  }{k^2}
	\right]
	\sum_{ \mathbf{p} + \mathbf{q} = \mathbf{k}}
	u_j(\mathbf{p} ) u_m(\mathbf{p} ) 
	- \nu k^2 u_i(\mathbf{k}) +
	\left[ 
	\delta_{ij} - \frac{k_i k_j }{k^2} 
	\right]
	f_j(\mathbf{k}) .
\end{equation*}

Notice the appearance of the operator \(  \delta_{ij} - k_i k_j  / k^2 \) in both the
term coupling the velocities in different modes, and in the external force. In both cases,
\(  \delta_{ij} \) represents a straight correspondence, while \(  - k_i k_j  / k^2 \)
is a term brought about by the pressure.

