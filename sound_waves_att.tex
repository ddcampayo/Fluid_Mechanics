\section{Attenuation of sound waves}
\label{sec:sound_waves_att}

This section contains a simple discussion of the role of viscosity in
the attenuation of sound waves. Original work is by Stokes, 1845.

Let us start with the Navier-Stokes momentum equation
\ref{eq:NS_Newtonian},
which we linearize by are disregarding high
order deviations from equilibrium values:
\begin{align}
  & \frac{\partial\rho' }{\partial t}  + \rho_0 \divu =0 \\
  & \label{eq:NS_Newtonian_const_visc}
    \rho_0 \frac{\partial \bfu }{\partial t} =
    - c^2 \nabla \rho' + \mu \nabla^2 \bfu
    + ( \mu + \lambda ) \nabla\cdot ( \divu ) .
\end{align}
The same assumptions as those in
Eqs. \ref{eq:sound_small_u}-\ref{eq:sound_small_p} have been made.
Also, we write $\kappa = c^2$ for the proportinality of pressure
modulations with density modulations, where $c$ is the sound speed
when viscosity is neglected.

The two viscosity coefficients are at their equilibrium values, since
any departure would entail higher order terms (they always multiply
the velocity). The subscript ``$_0$'' is not written on them for
simplicity.

Eq. \ref{eq:NS_Newtonian_const_visc} may seem different from previous
expressions, but it is the correct one when $\mu$ is constant, but
incompressibility does not hold. In this case, the term in
\ref{eq:epsilon_term_with_divergence} is not zero, but rather gives
rise to a $\mu \nabla\cdot ( \divu )$ term.

Our systems is not as simple as the one in Section
\ref{sec:sound_waves}, where the continuity and momentum equations
where equally simple. There, we could easily eliminate either the
velocity or the density from our equations in order to get a single
equation for the other field. Now, the continuity equation is the
same, but the one for momentum one is much more involved. This leaves
us with eliminating the density as the only viable scheme.

If we apply the gradient operator to continuity, and differentiate
with respect to space in momentum, we arrive to this equation for
the velocity only:
\[
  \frac{\partial^2 }{\partial t^2 } \mathbf{u} =
  c^2\nabla^2 \bfu + \nu \nabla^2\frac{\partial\bfu}{\partial t} +
  \zeta \nabla (\nabla\cdot \frac{\partial\bfu}{\partial t} ) ,
\]
where $\nu=\mu/\rho_0$, the usual kinematic viscosity, and we define
an additional kinematic bulk viscosity $\zeta=(\mu+\lambda)/\rho_0$.

Clearly, this reduces to our previous sound wave equation if there was
no viscosity.

Let us try a harmonic wave solution of the form
\[
  \bfu = \bfu_0 e^{i(\omega t - \beta x)} .
\]

Notice the usage of complex exponentials, which is common in wave
physics. Of course, the actual solution is real, so usually the real
part of the complex solution is kept (sometimes it is the imaginary,
which is equally valid). The flexibility comes partly from the fact
that $\mathbf{u}_0$ could be complex, which would represent a phase.
In our case, this is not important, but rather the possibility that
$\beta$ may be complex. In this case, its values encapsulates both a
real wave number and an spatial decay. Indeed, if
\[
  \beta = k - \alpha i ,
\]
then
\[
  \bfu = \bfu_0e^{- \alpha x} e^{i(\omega t - k x)} ,
\]
which clearly identifies $ \omega=2\pi /\lambda$ as the (real)
wave-number for a wave-lenght $ \lambda$, and $ \alpha=1 /L $, as a
sound attenuation coefficient, with $ L $ the penetration length.

Now, second time derivatives just yield
\[
  \frac{\partial^2  }{\partial t^2 } \mathbf{u} = -\alpha^2 \mathbf{u}
\]

But, it is quite interesting that these two second space derivatives
are not quite the same:

\[ \nabla^2 \mathbf{u} = -k^2 \mathbf{u}\]

\[ \nabla (\nabla\cdot \mathbf{u} ) = -k^2
(\mathbf{u}\cdot\mathbf{i})\mathbf{i} ,\]

where $ \mathbf{i} $ is the unit vector in the $x$ direction. Clearly,
the second version of the derivative produces a longitudinal wave,
with a vector component just along the direction of propagation!


Transverse waves


I know, sound waves are longitudinal. But, what happens if we plug
these derivatives for the $y$ Cartesian component. Well:

\[ -\omega^2 = - i \nu \omega \beta^2,\]

or

\[ \beta^2 = \frac{\omega}{i \nu} = - \frac{\omega i}{\nu} .\]

Now, this is a very classic complex analysis problem. Recall $
-i=e^{3\pi/2+2\pi} $ You do need the $ 2\pi$ to get the two
solutions. On of the solutions has the signs reversed, and corresponds
to a wave propagating and attenuating in the $-x$ direction. The one
we are looking for is:

\[ \beta =\sqrt{ \frac{\omega}{ \nu}}e^{ 7 \pi/4}.\]

Therefore

\[ k=\alpha =\sqrt{ \frac{\omega}{ 2 \nu}}.\]

This is an interesting wave, since the attenuation coefficient is
equal to the wave number. It looks like a simple function: $
e^{-x}\sin(x)$, which would be called simplistic by a physicist. If
you plot it you'll see it decays very fast, with just one maximum of
minimum of importance. (You can just type ``e\^-x*cos(x) from 0 to 10''
in google, it will plot it!). So, yes, sound waves are basically
longitudinal, since their transverse components get attenuated very
fast. How fast? As we said before, the attenuation length is the
inverse of $ \alpha$, so $ L=\sqrt{2\nu/\omega}$. This means that for
everyday "sound", i.e. audible frequencies, that length is quite
small. With the numerical data in the Table, that length will be only
be about 0.7 mm for a very low frequency of 10 Hz (just below the
hearing range), and will decrease as the inverse of the square root of
the frequency.

\begin{table}
\begin{tabular}{|c|c|c|c|c|}
  \hline
    & $ c$ (m/s) &  $ \nu$ (m$^2$/s) & $\zeta/\rho_0$ (m$^2$/s) & $ f_c$ (GHz)\\
  \hline
  \hline
  water & $1480$& $ 1.0 \times 10^{-6}$ & $ 3.1 \times 10^{-6}$ & $79$ \\
  \hline
  air & $340$   & $ 1.48\times 10^{-5}$ & $ 5 \times 10^{-6}$ ? & $0.7$ \\
  \hline
\end{tabular}
\caption{Numerical values for two important substances. Question marks
  are speculative, since in Cramer air is said to have a negligible
  bulk viscosity... but in a graph this quantity's value is seen to be
  about half the shear viscosity value, and air is mostly nitrogen.}
\end{table}

Longitudinal waves

Plugging the derivatives for the $x$ Cartesian component:

\[ -\omega^2 = - c^2\beta^2 - i (4\nu/3+\zeta/\rho_0)  \omega \beta^2.\]

We see that the viscosity now features the bulk viscosity, as seems
fitting for a longitudinal disturbance, which involves
compression. Also, a new important term appears. If viscosity were
negligible, the solution is just

\[  \omega =  c \beta , \]

the usual dispersion relation for sound. In general, though:

\[ \beta^2 =
\frac{\omega^2}{c^2+i(4\nu/3+\zeta/\rho_0)\omega}=\frac{\omega^2}{c^2}\frac{1}{1+
  i\omega/\omega_c},\]

where we define the important crossover angular frequency

\[ \omega_c := \frac{c^2}{4\nu/3+\zeta/\rho_0}.\]

Numerical values can be found in the table (we provide the linear
frequency). The value is really high for hearing range, which goes up
to about 20 kHz for humans, 160kHz for porpoises, which hold the
record. Medical ultrasound goes as high as 16 MHz only, and only
acoustic microscopy reaches a few GHz, the range at which our value
for air sits.


Low frequencies


At frequencies much below the
crossover frequency, we may expand the term in the denominator in a
Taylor series, then again for the square root. The end result is

\[ \beta = \frac{\omega}{c} (1-i   \omega/(2 \omega_c)).\]

 

Therefore the wave number is

\[ \beta = \frac{\omega}{c},\]

as if there were no viscosity. The attenuation coefficient is

\[ \alpha = \frac{\omega^2}{2 c\omega_c}=\frac{\omega^2}{2 c^3}(4\nu/3+\zeta/\rho_0). \]

It therefore grows as the square of the frequency.  This agrees with
the expression in Stokes' law of sound at wikipedia (but for a
factor of two in the general formula, which I have just corrected -
let's see if it stays). This would mean that for that very high
porpoise's pitch at 160Hz the attenuation length would be about 1.5
kilometers (in water, of course), which may be important for
long-range communication of these animals. For medical ultrasound at
16MHz this length is 14cm, which can clearly have an impact for human
tissues (I am not sure if this attenuation may be used to our
advantage). For the human bodies I have used water values, which is a
fair approximation.


High frequencies


At frequencies much
higher than the crossover, we may neglect the "1" in the denominator,
to obtain

\[ \beta^2 = \frac{\omega\omega_c}{i c^2} .\]

Now, this looks familiar, especially if we substitute the crossover
frequency:

\[ \beta^2 = \frac{\omega}{i(4\nu/3+\zeta/\rho_0)} .\]

Basically, the same expression as for transverse waves, with a
different combination of viscosities. The conclusion is similar:

\[ k=\alpha =\sqrt{ \frac{\omega}{ 2(4\nu/3+\zeta/\rho_0)}},\]

and the resulting waves are heavily attenuated.



The crossover

A fast check on the above approximations is to see what happens when
the frequency is exactly the crossover frequency. At this point, the
growth of the attenuation coefficient as the square of frequency
crosses over to a growth as the square root. This would mean a bend in
a log-log plot, between two straight lines with different slopes.

The extrapolation of the low frequency expression yields

\[ \alpha_c\approx \omega_c/(2 c) , \]

whereas the high frequency expression yields

\[ \alpha_c\approx \omega_c/(\sqrt{2} c) , \]

not so different at all.

The exact expression can be shown to be, after some complex algebra,

\[ \alpha_c\approx \omega_c \sin(\pi/8) / (2^{1/4} c) , \]

a value just a bit below the other two. This means the approximations
remain quite fair up to the limit of their respective ranges.

By the way, for water this value corresponds to an attenuation length
of about 58 nm, which is really short. For air, it is about 1.5
micrometers, the size of a really small cell.



 
