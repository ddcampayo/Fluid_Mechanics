\chapter*{}

\vspace{6ex}
%\begin{quotation}
\begin{quote}
  \greektext

  Ποταμο˜ισι το˜ισιν αὐτο˜ισιν ἐμβαίνουσιν, ἕτερα καὶ ἕτερα ὕδατα ἐπιρρε˜ι  \\%[0.5cm]
  \latintext
  \begin{em}
    Ever-newer waters flow on those who step into the same rivers.  \\%[1cm]
  \end{em}
  Heraclitus \cite{Diels-Kranz}.%, Fragmente der Vorsokratiker, 22 B12
\end{quote}

%\end{quotation}

\newpage

\chapter*{Foreword}


Just a compilation of notes on different aspects of fluid
dynamics I have collected over the years.

The final push was from teaching the course Fluid Dynamics on the
International Master of Nuclear Fusion, course 2018. I realized much
material was scattered all over the place, on notes, articles, blog
entries \ldots

I may upload these notes to a collaborative site (e.g. github) so that
other people may contribute. As of today, I am the sole author.\\[4cm]

Daniel Duque \\
CEHINAV (Naval Model Basin Research Group) \\
Universidad Polit\'ecnica de Madrid \\
Madrid, Spain 2018


\chapter{Introduction}

Fluids, by definition, flow.

The main field is the velocity field, and the displacement is
secundary \ldots


\section{Notation}
\label{sec:notation}

When labeling coordinates, we will follow this convention:
\begin{equation*}
  \bfr =
  \begin{cases}
   ( x, y, z)  &\qquad \text{Cartesian} \\
   ( r, \theta, 0 ) & \qquad \text{Polar} \\
   ( \rho, \theta, z) &\qquad \text{Cylindrical} \\
    ( r , \phi , \theta) &\qquad \text{Spherical} .
  \end{cases}
\end{equation*}
For spherical coordinates, we stick to the naming convention that is
most common in physics, and is specified by ISO standard 80000-2:2009,
and earlier in ISO 31-11 (1992). I.e.: $\phi$ is the azimuthal angle,
about the $z$ axis, and $\theta$ is the polar angle with the $z$ angle
(also called ``the co-latitude''.)

In cylindrical coordinates, $\rho$, the distance to the $z$ axis,
could be confused with the density, but it always is quite clear to
tell them apart. We deviate from the standard (ISO 31-11, to be
precise) by using $\theta$ for the azimuthal angle for 2D polar and
cylindrical coordinates. We are aware that the right symbol is $\phi$
(or, $\varphi$), and we are moving towards implementing the
recommendations, but the usage of $\theta$ is quite ingrained. For the
2D polar distance to the origin, either $r$ or $\rho$ are valid,
depending on whether this 2D case is taken as: $z = 0$ from
cylindrical coordinates; or, as $\theta \to \pi/2$ from spherical
ones. We choose the former.

The velocity field is represented in many ways. Here, we use bold
letters for vectors, and italice letters with subindices for its
components, Cartesian or otherwise:
\begin{align*}
  \bfu =
  \begin{cases}
     ( u_x, u_y, u_z)& \qquad \text{Cartesian} \\
     ( u_r, u_\theta, 0 )& \qquad \text{Polar} \\
     ( u_\rho, u_\theta, u_z)& \qquad \text{Cylindrical} \\
     ( u_r , u_\phi , u_\theta)& \qquad \text{Spherical} .
   \end{cases}
\end{align*}
This convention is applied for any vector field, we just write it down
explicitely because the velocity is, by far, the most used field in
these notes.

We \emph{do not} use the traditional convention of using different
letters for each Cartesian velocity component:
\[
  \bfu = ( u , v, w ) =
  u \bhe_x +
  v \bhe_x +
  w \bhe_x   \qquad \text{(not used)} 
\]
(we do use pointy hats, or ``circumflexes'', for unit vectors.)

\section{Style}

In these notes, concepts are introduced \emph{at needed}. This is at
variance with other texts, in which there may be an introduction to
the general physics of fluids, another for mathematical analysis, and
so on. So, for example, the concept of streamlines is not introduced
until the potential flow past a cylinder is discussed. In fact, only
the simple 2D case is introduced --- the more complicated axisymmetric
situation is later described, for the flow past a sphere.

The few starred sections refer to ``advanced'' material, some of which
may be the authors' own calculations (therefore, to be read with
care).
