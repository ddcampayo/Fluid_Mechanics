\chapter*{}

\vspace{6ex}
%\begin{quotation}
\begin{quote}
  \greektext

  Ποταμο˜ισι το˜ισιν αὐτο˜ισιν ἐμβαίνουσιν, ἕτερα καὶ ἕτερα ὕδατα ἐπιρρε˜ι  \\%[0.5cm]
  \latintext
  \begin{em}
    Ever-newer waters flow on those who step into the same rivers.  \\%[1cm]
  \end{em}
  Heraclitus \cite{Diels-Kranz}.%, Fragmente der Vorsokratiker, 22 B12
\end{quote}

%\end{quotation}

\newpage

\chapter*{Foreword}


Just a compilation of notes on different aspects of fluid
dynamics I have collected over the years.

The final push was from teaching the course Fluid Dynamics on the
International Master of Nuclear Fusion, course 2018. I realized much
material was scattered all over the place, on notes, articles, blog
entries \ldots

I may upload these notes to a collaborative site (e.g. github) so that
other people may contribute. As of today, I am the sole author.\\[4cm]

Daniel Duque \\
CEHINAV (Naval Model Basin Research Group) \\
Universidad Polit\'ecnica de Madrid \\
Madrid, Spain 2018


\chapter{Introduction}

Fluids, by definition, flow.

The main field is the velocity field, and the displacement is
secundary.


\section{Notation}

\section{Style}

In these notes, concepts are introduced \emph{at needed}. This is at
variance with other texts, in which there may be an introduction to
the general physics of fluids, another for mathematical analysis, and
so on. So, for example, the concept of streamlines is not introduced
until the potential flow past a cylinder is discussed. In fact, only
the simple 2D case is introduced --- the more complicated axisymmetric
situation is later described, for the flow past a sphere.

